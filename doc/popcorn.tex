\documentclass[titlepage,10pt]{article}
\setlength{\oddsidemargin}{0cm}
\setlength{\evensidemargin}{0cm}
\setlength{\topmargin}{.5in}
\setlength{\headheight}{0cm}
\setlength{\headsep}{0cm}
\setlength{\textwidth}{6.5in}
\setlength{\textheight}{9in}
\setlength{\parindent}{0cm}
\setlength{\parskip}{6pt}
\usepackage{amsmath}
%\usepackage{makeindex}
%\makeindex
%\renewcommand\contentsname{Table of Contents}
%\numberwithin{subsubsection}{subsection}



\begin{document}

\title{Popcorn Reference Manual \\ \textsc{DRAFT}}
\date{November 1999}
\author{Greg Morrisett et al.}

\maketitle

\tableofcontents

\newpage

%%%<ul>
%%%  <li><a href="#overview">Overview</a> </li>
%%%  <li><a href="#examples">Examples</a></li>
%%%  <li><a href="#language">Language</a></li>
%%%  <ul>
%%%    <li><a href="#language">Pre-processor, namespaces, lexical conventions</a> </li>
%%%    <li><a href="#types">Types</a></li>
%%%    <li><a href="#declarations">Declarations</a></li>
%%%    <li><a href="#expressions">Expressions</a></li>
%%%    <li><a href="#statements">Statements</a></li>
%%%    <li><a href="#programs">Programs</a></li>
%%%  </ul>
%%%  <li><a href="#library">Library</a></li>
%%%  <li><a href="#tools">Tools</a></li>
%%%  <ul>
%%%    <li> <a href="#lex">popocamllex</a></li>
%%%    <li> <a href="#bison">bison</a></li>
%%%  </ul>
%%%</ul>

\section{Overview\label{overview}}

Popcorn is a research prototype programming language that we (and
others) are using to explore various issues in type-safe language
design, type-directed compilation, proof carrying code, safe runtime
code generation, and other issues that arise in modern language design
and implementation.  Syntactically, Popcorn resembles a mix of C, C++,
and Java and most of the control constructs and scoping mechanisms are
lifted directly from these languages.  With respect to type and type
structure, Popcorn is more akin to the ML family of languages, in that
it provides support for parametric polymorphism, some type inference,
algebraic datatypes with pattern matching, etc.  In practice, it is
relatively straightforward to port C code to Popcorn.   Porting ML code
is also straightforward though the syntax and emphasis on imperative
aspects makes this a bit more involved.  In the near future, we are
planning support for various kinds of objects, classes and subtyping,
perhaps along the lines of LOOM [?], Cecil [?], or Moby [?] with the
goal of being able to port code from C++, Java, or these OO-based
languages relatively easily. 

Smoothly integrating all of these facilities is extremely difficult and
we view Popcorn as a vehicle for exploring tradeoffs.  Consequently, we
expect the definition of the language to evolve quite rapidly over time.
To limit the scope of exploration, we have a determined a set of (loose)
goals that we wish to follow as the language evolves.  In particular,
Popcorn should be: 

\begin{description}

\item [Type Safe] The benefits of type safety are overwhelming (we take
this as an axiom and the prime directive).  We believe that advanced,
static type systems provide the best linguistic mechanism to build and
organize the abstractions needed to for large, reliable, secure, and
high-performance systems.

\item [Close to C] Though this goal seems in direct conflict with the
previous goal, our aim is to make it easy for C programmers to port or
write Popcorn code.  As such, we hope for the Popcorn type system to
enable more idioms from C as long as they continue to be type-safe.
Already, we have ported some tools (e.g., bison) from C to Popcorn and
have found this to be relatively easy.  A major long-term goal is to
make it easy to reason about the run-time costs of Popcorn code -- a
programmer should be able to determine the big-O space and time
complexity of Popcorn code relatively easy.  Currently, this is not the
case because of the reliance on a (conservative) garbage collector but
part of our research is aimed at getting rid of, or replacing the
collector with some more explicit mechanism without sacrificing type
safety.

\item [Easy to Interface] One of the best things about C code is that it
is relatively easy to interface it to existing standards (including
libraries written in C!).  By standards, I mean things like COM or
CORBA, device drivers, Tk, Unix, Win32, sockets etc.  Most if not all of
these ``standard tools'' provide interfaces that are easy to access
using C type definitions and function calls.   Popcorn should also make
it easy to work with and use these existing tools to the best degree
possible.  In contrast, type-safe languages such as Java, ML, or Scheme
make it hard to interface to such facilities without writing a lot of
``glue'' and often this glue must be written in a language like C.  In
part, this is because such languages do not provide explicit facilities
for invoking C code, and in part because they do not provide facilities
for defining and manipulating a rich set of data representations.

\item [Easy to Compile] Our research is primarily in type systems, not
advanced compiler optimizations.  As such, we wish to take advantage of
compilation techniques that have been or are being developed by the rest
of the language and compiler communities.  If by changing the Popcorn
language, we can make it easier to generate good, high-performance code,
then we should as long as the change does not conflict with the other
goals.

\end{description}


As it stands, Popcorn fails to achieve all of these goals in a nice way
(thankfully -- that's why there's work to do!) but goes pretty far in
comparison to other languages.   As such, we think it can be used today
to build interesting systems and would encourage anyone to try it and
give us feedback about the design and/or implementation.   Some good
questions for us to answer are: 


\begin{description}

\item [Why shouldn't you just use Java?] Though Java shares with Popcorn
many syntactic and control-flow constructs, the type system of Java is
relatively weak.   There is no support for parametric or F-bounded
polymorphism, sub-typing is confused with sub-classing, there is little
or no support for modules, algebraic datatypes, etc.  Though we think
Java is a relatively clean and simple design, Popcorn allows us to
explore alternatives that we think are equally interesting and useful.
Furthermore, compiling Java to good code is extremely difficult because
the memory model is so constrained, and the synchronization/thread model
is in our opinion fairly poor.

\item [Why shouldn't you just use ML?] Though ML provides many of the
same typing facilities as Popcorn (and in some ways more), the syntax
and shift to a functional paradigm have, in my opinion, alienated many
programmers who would otherwise enjoy such a language.  Interfacing ML
systems to C libraries and code is do-able, but not completely
transparent.  Compiling ML to good code is also doable, but in my
opinion, requires a much more sophisticated compiler.  I think the Ocaml
project at Inria has demonstrated the benefits of evolving a language
rapidly to incoporate facilities that, while of little interest to the
language theory community, in practice, provide real benefit to users.
So, to a large degree, we are hoping to emulate and provide some of the
benefits of the Ocaml project but in the context of a (largely)
imperative language.

\item [Why shouldn't you just use Modula-3 or Eiffel?] Part of the issue
is a simple one of syntax -- it's relatively easy for a C, C++, or Java
programmer to pick up the syntax of Popcorn and get going immediately.
Part of the issue is library design -- again, it's easy to use existing
libraries written in C.  And part of the issue is a desire to further
explore typing issues that are not covered by these languages.

\item [Why shouldn't you just use C?] C is no where near type safe.
Though it is possible to insert lots of extra dynamic checks, and lots
of extra state to achieve an implementation that allows standard C code
to execute, the truth of the matter is that C is a lousy language for
building abstractions because it provides no mechanism for enforcing
abstractions.  You basically get functions and blocks of memory and
that's about it.

\end{description}

This document is meant to give you a brief overview of the syntax,
type-system, dynamic semantics, and tools that currently make up the
Popcorn language.  We have tried to detail the strengths and weaknesses
of the language here and to point out areas where we expect the language
to improve (or at least change in some substantial way.)  We would
appreciate any criticism, comments, suggestions, etc. from users.

\subsection{Popcorn and Typed Assembly Language}

Originally, Popcorn was developed as a vehicle for showing that
type-safe, high level languages could be compiled to \textit{Typed
Assembly Language} (TAL).   Thus, to understand some of the design
issues of Popcorn, it is useful to know a little bit about TAL.

Like the Java Virtual Machine Language (JVML), TAL is a
\textit{statically typed} low-level language intended to serve as the
output of a compiler for a type-safe high-level language.  Such
languages are extremely useful in security contexts where we wish to
ensure that a number of ``bad'' things cannot happen when code is
executed.  In particular, type-safety (in the TAL sense) at least
implies address space isolation, so it is not necessary to use
hardware-enforced address translation to ensure that a TAL program does
not read or write locations in memory that are not in its conceptual
address space.  As with JVML, the type system of TAL ensures many
stronger safety properties, including the fact that ``private'' fields
of data structures cannot be read or written, all functions are passed
the right number (and type) of arguments, stack frames from a caller are
protected from a callee reading or writing local variables, etc.

But unlike the JVML, TAL instructions correspond to real Intel 80x86
instructions instead of virtual machine instructions. Consequently,
there is no overhead for interpretation, and we can use a fully
optimizing compiler to produce our TAL code, but still be able to
type-check.  Indeed, for critical loops, it is possible to code directly
in TAL to achieve the best possible performance (subject to the
limitations of the type system.)   Thus, TAL seems like a more
attractive target for systems such as the PLAN Active Networks project
at the University of Pennsylvania [?] that need safe but highly
efficient code to achieve their goals.

Furthermore, the TAL type system provides a set of general type
constructors which may be used to encode typing abstractions for a
number of type-safe, high-level languages such as SML, Scheme, or
Modula-3.  In contrast, the JVML type system is closely related to the
Java source language and is not very well suited to compiling other
languages.  For instance, it is not possible to eliminate tail-calls to
another method using JVML, but it is trivial to do so in TAL.  This is
because the JVML design bakes in a fixed notion of method, calling
conventions, etc., whereas TAL provides a set of general-purpose type
constructors that can be used to build custom procedural abstractions,
custom calling conventions, etc.  As another example, it is difficult to
compile high-level languages such as SML that provide parametric
polymorphism or higher-order type constructors to JVML because the type
system of JVML does not have such facilities.  Rather, a compiler must
insert additional run-time checks (downcasts) to convince the JVML
static verifier to pass the code, and these run-time checks can hurt
performance.  Again, because TAL provides support for features like
parametric polymoprhism, it is not necessary to insert run-time checks
to get such code past the verifier.

Popcorn was designed as a relatively simple, type-safe language that
could be easily compiled to TAL in order to demonstrate all of these
claims about the advantages of TAL.  For more information on TAL, please
refer to the following publications:

\begin{itemize}

\item Greg Morrisett, Karl Crary, Neal Glew, Dan Grossman, Richard
Samuels, Frederick Smith, David Walker, Stephanie Weirich, and Steve
Zdancewic, \textbf{TALx86: A Realistic Typed Assembly Language}
submitted to the \textit{1999 ACM SIGPLAN Workshop on Compiler Support
for System Software}.

    \begin{itemize}
    \item abstract: $<$http://www.cs.cornell.edu/talc/papers/talx86-wcsss-abstract.html$>$
    \item dvi: $<$http://www.cs.cornell.edu/talc/papers/talx86-wcsss.dvi$>$
    \item pdf: $<$http://www.cs.cornell.edu/talc/papers/talx86-wcsss.pdf$>$
    \end{itemize}

\item Neal Glew and Greg Morrisett. \textbf{Type-Safe Linking and
Modular Assembly Language} \textit{Twenty-Sixth Symposium on Principles
of Programming Languages (POPL '99),} pages 250-261, San Antonio,
January 1999.

    \begin{itemize}
    \item abstract: $<$http://www.cs.cornell.edu/talc/papers/mtal-abstract.html$>$
    \item dvi: $<$http://www.cs.cornell.edu/talc/papers/mtal.dvi$>$
    \item pdf: $<$http://www.cs.cornell.edu/talc/papers/mtal.pdf$>$
    \end{itemize}

\item Karl Crary, David Walker, and Greg Morrisett. \textbf{Typed Memory
Management in a Calculus of Capabilities.} \textit{Twenty-Sixth
Symposium on Principles of Programming Languages (POPL '99), }pages
262-275, San Antonio, January 1999.

    \begin{itemize}
    \item abstract: $<$http://www.cs.cornell.edu/talc/papers/capabilities-abstract.html$>$
    \item dvi: $<$http://www.cs.cornell.edu/talc/papers/capabilities.ps.gz$>$
    \item pdf: $<$http://www.cs.cornell.edu/talc/papers/capabilities.dvi$>$
    \end{itemize}

\item Karl Crary, Stephanie Weirich, and Greg Morrisett.
\textbf{Intensional Polymorphism in Type-Erasure Semantics.}
\textit{1998 International Conference on Functional Programming,} pages
301-312, Baltimore, September 1998.

    \begin{itemize}
    \item abstract: $<$http://www.cs.cornell.edu/talc/papers/typepass-abstract.html$>$
    \item dvi: $<$http://www.cs.cornell.edu/talc/papers/typepass.ps.gz$>$
    \item pdf: $<$http://www.cs.cornell.edu/talc/papers/typepass.dvi$>$
    \end{itemize}

\item Greg Morrisett, Karl Crary, Neal Glew, and David Walker.
\textbf{Stack-Based Typed Assembly Language.} \textit{1998 Workshop on
Types in Compilation.} Published in Xavier Leroy and Atsushi Ohori,
editors, \textit{Lecture Notes in Computer Science}, volume 1473, pages
28-52. Springer-Verlag, 1998.

    \begin{itemize}
    \item abstract: $<$http://www.cs.cornell.edu/talc/papers/stal-abstract.html$>$
    \item dvi: $<$http://www.cs.cornell.edu/talc/papers/stal.ps.gz$>$
    \item pdf: $<$http://www.cs.cornell.edu/talc/papers/stal.dvi$>$
    \end{itemize}

\item Greg Morrisett, David Walker, Karl Crary, and Neal Glew.
\textbf{From System F to Typed Assembly Language.} In
\textit{Twenty-Fifth ACM Symposium on Principles of Programming
Languages,} pages 85-97, San Diego, January 1998.

    \begin{itemize}
    \item abstract: $<$http://www.cs.cornell.edu/talc/papers/tal-abstract.html$>$
    \item dvi: $<$http://www.cs.cornell.edu/talc/papers/tal.ps.gz$>$
    \item pdf: $<$http://www.cs.cornell.edu/talc/papers/tal.dvi$>$
    \end{itemize}

\end{itemize}

or visit the TAL project homepage at $<$http://www.cs.cornell.edu/talc$>$.


\section{Some Popcorn Examples\label{examples}}


\section{The Popcorn Language Reference\label{language}}


\subsection{Pre-processor}

Popcorn's syntax is (grudgingly) designed so that it passes through the
C pre-processor without confusion.  We think of the pre-processor as
separate from the Popcorn language, but in practice the Popcorn compiler
first passes source files through the C pre-processor.  Thus we provide
the same conveniences (and pitfalls!) provided by the macro facilities
of C.  Large Popcorn programs generally use header files and
\texttt{\#include} directives much as in C.  There are some important
differences and caveats, however: 

\begin{itemize}

\item In C, a type declaration (such as a struct) may be made multiple
times.  This is not true in Popcorn.  Hence a struct declaration in a
header file will generally have the \texttt{extern} qualifier whereas
the coresponding implementation file must have the same declaration
without the qualifier.  The header file should \textit{not} be included
by the corresponding implementation file, even indirectly.

\item Because the ``semantics'' of \#include is textual substitution, it
is unwise to the semicolon forms of prefix and open in a header file.
These forms extend their scope over the rest of the file, where ``rest
of the file'' refers to the file \textit{after} pre-processing.

\end{itemize}

Given these differences, the standard idiom for a header file looks like this:

%-%\texttt{
%-%\begin{tabular}{|l|} \hline 
%-%\#ifndef FOO{\_}H
%-%\#define FOO{\_}H
%-%
%-%// other includes here, for example:
%-%\#include "bar.h"
%-%
%-%prefix Foo \{
%-%open   Foo \{
%-%  // declarations for public members of the Foo namespace here
%-%\}
%-%\}
%-%\#endif
%-%\hline
%-%\end{tabular}
%-%}

\begin{verbatim}
#ifndef FOO_H 
#define FOO_H 

// other includes here, for example: 
#include "bar.h" 

prefix Foo { 
open   Foo { 
  // declarations for public members of the Foo namespace here 
} 
} 
#endif
\end{verbatim}

Then the implementation of the Foo namespace could be in a file foo.pop
with this idiom:

%-%\texttt{
%-%\begin{tabular}{|l|} \hline
%-%// this prevents foo.h from being included here, even if other header
%-%files include it: \\
%-%\#define FOO{\_}H \\
%-%\\
%-%// other includes, for example: \\
%-%\#include ``bar.h'' \\
%-%\\
%-%prefix Foo; // semicolon form okay because we won't include this file
%-%elsewhere \\
%-%open Foo; \\
%-%\\
%-%// implementation here \\
%-%\hline
%-%\end{tabular}
%-%}

\begin{verbatim}
// this prevents foo.h from being included here, even if other 
// header files
include it: 
#define FOO_H 

// other includes, for example: 
#include "bar.h" 

prefix Foo; // semicolon form okay because we won't include this file
elsewhere 
open Foo; 

// implementation here
\end{verbatim}

Note that if there are not cycles among header files, then the
\texttt{\#define} in the implementation file is unncessary.


\subsection{Lexical Conventions}

To do:  similar to C and Java.  Key thing to note, comments can be
either C-style (/* ... */) or C++/Java style (// ...). Upper/lower case
is significant, etc.

Also note certain name conflicts with the runtime (for example,
\texttt{GC{\_}Malloc}) are not caught until link-time.

\textbf{Reserved Words}:

%-%\texttt{\textbf{
%-%abstract abstype array bool break byte case catch char chr \newline
%-%const continue default do else exn exception extern finally \newline
%-%for fprintf handle if int new new{\_}array new{\_}string null \newline
%-%open ord prefix printf public raise return short signed size \newline
%-%sprintf static string struct switch try union unsigned void while with
%-%}}

\begin{verbatim}
abstract abstype array bool break byte case catch char chr 
const continue default do else exn exception extern finally 
for fprintf handle if int new new_array new_string null open
ord prefix printf public raise return short signed size 
sprintf static string struct switch try union unsigned void 
while with
\end{verbatim}


\subsection{Namespaces}

To do:  similar to C++.  Use Foo::bar where Foo is the ``name space''
(think structure or module).  You can write prefix Foo \{
$<$top-level-decls$>$ \} to cause the Foo:: to be implicitly appended to the
front of identifiers defined within the braces.  You can write open Foo
\{ ... \} in a context where you don't want to have to write Foo::bar
all the time and just want to write bar.   The forms prefix Foo; and
open Foo; are the same except that the rest of the file is considered to
be ``within the braces.''   Scoping and shadowing at the namespace level
works as follows.  If an identifier \texttt{bar} is in scope without any
namespace prefix, then that binding is used.  Else the innermost opened
namespace that defines \texttt{bar} is used to bind the identifier.


\subsection{Type Expressions\label{types}}

Note: We will soon go to a Java-style definite assignment rule in which
case the ``default values'' described in this section are no longer
relevant.

Popcorn provides a number of primitive types and type constructors
similar to that of ML.  In particular, the language defines a void type
(similar to ML's unit type except there is no value of type void),
various integer types (signed and unsigned), string and array types,
(polymorphic) function types, (polymorphic) struct and union types,
abstract types, and exception types.  In this section, we detail the
syntax and informal semantics of type expressions and declarations.

%-%\texttt{
%-%\begin{tabular}{lrl}
%-%<type-exp>      & ::= & <prim-type> <type-modifiers> \\
%-%<prim-type>     & ::= & \parbox[t]{5in}{
%-%                        <type-var> | \textbf{void} | <integer-type> | 
%-%                        \textbf{bool} | \textbf{char} | \textbf{string}
%-%                        | *(<type-exp>,...,<type-exp>) | \textbf{exn} |
%-%                        [<<type-exp>,...,<type-exp>>]<id> |
%-%                        <type{\_}exp>\textbf{array} } \\
%-%<type-var>      & ::= & [a-zA-Z{\_}][a-zA-Z0-9{\_}]* \\
%-%<integer-type>  & ::= & \parbox[t]{5in}{
%-%			[\textbf{unsigned}] \textbf{int} |
%-%                        [\textbf{unsigned}] \textbf{short} |
%-%                        [\textbf{unsigned}] \textbf{byte} } \\
%-%<type-modifiers> & ::= & \parbox[t]{5in}{
%-%			<empty> | [] <type-modifiers> |
%-%                        <fntype-modifier> <type-modifiers> } \\
%-%<fntype-modifier> & ::= & \parbox[t]{5in}{
%-%			[<id>][<<type-var>,...,<type-var>>]
%-%                        (<type-exp>,...,<type-exp>) }
%-%\end{tabular}
%-%}

\begin{verbatim}
<type-exp> ::= <prim-type> <type-modifiers>
<prim-type> ::= <type-var> | void | <integer-type> | bool | char | 
                string | *(<type-exp>,...,<type-exp>) | exn | 
                [<<type-exp>,...,<type-exp>>]<id> | <type_exp>array
<type-var> ::= [a-zA-Z_][a-zA-Z0-9_]*
<integer-type> ::= [unsigned] int | [unsigned] short | [unsigned] byte
<type-modifiers> ::= <empty> | [] <type-modifiers> |
                     <fntype-modifier> <type-modifiers>
<fntype-modifier> ::= [<id>][<<type-var>,...,<type-var>>](<type-exp>,...,<type-exp>)
\end{verbatim}

\subsubsection{Type Variables}

%-%\texttt{a, b, foo}

\begin{verbatim}
a, b, foo
\end{verbatim}

Type variables are written as identifiers.  Type variables are used
within polymorphic type expressions (e.g., the types of polymorphic
functions) or within declarations of polymorphic type constructors
(e.g., structs, unions, and abstypes).  Currently, not all types may be
used to instantiate a given type variable.  Intuitively, only types
whose size is a word (4 bytes) may be used to instantiate a type
variable, and void may not be used to instantiate a variable.

The type variable lists for function types, struct declarations, union
declarations, and abstype declarations are binding occurrences which
shadow type names already in scope, including struct and union names.
In practice this rarely occurs since programmers tend to use different
identifiers for type variables and type names.

\subsubsection{The Void Type}

%-%\texttt{\textbf{void}}

\begin{verbatim}
void
\end{verbatim}

The void type is used to indicate that a function returns no values or
that a union data constructor (or exception) carries no values.  It is
illegal to instantiate a type variable with void.  It is expected that
this restriction will be lifted at some point in the future.  There are
no values of type void.

\subsubsection{Integer Types}

%-%\texttt{
%-%[\textbf{unsigned}] \textbf{int} \newline
%-%[\textbf{unsigned}] \textbf{short} \newline
%-%[\textbf{unsigned}] \textbf{byte}
%-%}

\begin{verbatim}
[unsigned] int
[unsigned] short 
[unsigned] byte
\end{verbatim}

There are 6 types corresponding to integers:  int, unsigned int, short,
unsigned short, byte, and unsigned byte.  The int type represents a
signed, 32-bit integer value.  The short type represents a signed,
16-bit value, and the byte type represents a signed, 8-bit value.
Clearly, the unsigned qualifier represents the unsigned 32, 16, or 8-bit
value space.  We expect to add support for long and unsigned long
(64-bit integer values) as well as floats and doubles in the near
future.   Arithmetic on these types should work the same as under C, and
operators such as +,-,*, etc. are overloaded as appropriate.  Note that
the C semantics for performing arithmetic on mixed-size objects requires
that the values be cast up to the largest appropriate size.  See the C
reference manual for details.

Decimal-based integer constants are written as in C.  Popcorn also
supports hexadecimal (e.g., 0xfeedface), octal and binary.

The default initial value for variables of the integer types is 0 (zero).

\subsubsection{The Boolean Type}

%-%\texttt{\textbf{bool}}

\begin{verbatim}
bool
\end{verbatim}

The type bool represents the type of boolean values.  The constants true
and false are used to create boolean values.  Most comparison operations
(e.g., ``=='', ``$<$='', ``!='', etc.) return a boolean value and both
the conditional expression and conditional statement expect boolean
expressions as their first argument.

The default initial value for variables of type bool is false.

\subsubsection{The Character Type}

%-%\texttt{\textbf{char}}

\begin{verbatim}
char
\end{verbatim}

The type char is distinct from the integer types but one may convert
chars to and from integers via ord() and chr().  Currently, chars are
8-bit, ASCII representations of characters.  At some point, we may have
a separate type for unicode characters or change char to unicode.
Character literals are as in C (e.g., 'a', 'b', '$\backslash$n')
including escaped characters.  Switch expressions can perform a match on
character expressions and some simple arithmetic (may be?) is supported
on characters.

The default initial value for variables of type char is
'$\backslash$000'.

\subsubsection{The String Type}

%-%\texttt{\textbf{string}}

\begin{verbatim}
string
\end{verbatim}

Strings are like arrays of chars, but they are not the same.  Unlike
Java, they are mutable and unlike C or C++, strings are not
$\backslash$000 terminated and come with their own size.   String
subscript and update are written in a style analogous to array subscript
and update (e.g., s[i] = s[i+1]).  Operations on strings, as with
arrays, check that the index is in bounds before proceeding and raise
the ArrayBoundsExceptions exception if the index is out of bounds.
(Actually, this currently causes program termination, but this should
change soon.)  String literals are written as in C (e.g., "foobar",
"baz$\backslash$n") including escaped characters.  New
($\backslash$000-filled) strings may be created by using the
new{\_}string function defined in the core library.  The library also
provides other string manipulation functions similar to the C standard
library (e.g., strcmp, strcopy, etc.).

The string type is distinct from the type of character arrays.  Hence
none of the following are well-typed Popcorn statements: 

%-%\texttt{
%-%\begin{tabular}{l}
%-%string x   = \{'a', 'b'\}; \\
%-%char   y[] = ``ab''; \\
%-%x = y; 
%-%\end{tabular}
%-%}

\begin{verbatim}
string x   = {'a', 'b'};
char   y[] = "ab";
x = y;
\end{verbatim}

but the following is well-formed:

%-%\texttt{
%-%\begin{tabular}{l}
%-%string x   = ``ab''; \\
%-%char   y[] = \{'c', 'd'\}; \\
%-%x[0] = y[0];
%-%\end{tabular}
%-%}

\begin{verbatim}
string x   = "ab"; 
char   y[] = {'c', 'd'}; 
x[0] = y[0]; 
\end{verbatim}

The default initial value for variables of type string is the empty
string ("").

We expect to add more built-in operations on strings including some
forms of pattern matching.

\subsubsection{The Array Type}

%-%\texttt{
%-%\begin{tabular}{ll}
%-%\textbf{int}[]   & \textit{// an array of integers} \\
%-%\textbf{char}[]  & \textit{// an array of characters} \\
%-%\textbf{int}[][] & \textit{// an array of an array of integers} \\
%-%<\textbf{int}> \textbf{array} & \textit{// alternative syntax for array
%-%    types}
%-%\end{tabular}
%-%}

\begin{verbatim}
int[]   // an array of integers
char[]  // an array of characters
int[][] // an array of an array of integers
<int>array // alternative syntax for array types
\end{verbatim}


As in C, array types are declared by putting a pair of brackets ([])
after a type expression.   You can specify the size of an array in the
type, but this is discouraged and is only provided for ease of porting
C. Unlike C, you cannot specify multi-dimensional arrays directly.  In
this respect, the arrays of Popcorn are more like Java arrays.
Association of [] is to the left.  As a convenience and to avoid
confusion when combining array types with function types, Popcorn
provides the alternative syntax $<$t$>$array.

Arrays literals, as in C, are created using \{exp1,...,expn\}.  Empty
arrays (arrays of size 0) have a special syntax:  \{:$<$type-exp$>$\} so
that the type-checker can easily identify the type of the expression.
Array subscript and update are written as in C (e.g., x[i] = x[i+1]) and
the indices are always checked to see if they are in bounds.  Currently,
an out of bound index causes program termination, but in the near future
we expect to raise the pre-defined ArrayBoundsException exception
instead.  New arrays may be created by calling the polymorphic
new{\_}array function which takes an initial element of the appropriate
type and an integer size (greater than or equal to 0).  An array with
elements of an integer type may be created either with new{\_}array or
with new{\_}array4.  The latter does not require an initial element and
may be more efficient.

The default initial value for variables of type array is a 0-sized array
unless the size appears in the type.  In the latter case, the default
initial value is an array of the appropriate size with elements equal to
the default initial value of the element type.  If the element type has
no default initial value then neither does the array type with size
included.

\subsubsection{Tuple Types}

%-%\texttt{
%-%\begin{tabular}{ll}
%-%*(\textbf{int},\textbf{int}) & \textit{// the type of a pair of
%-%    integers} \\
%-%*(\textbf{int},\textbf{string},\textbf{char}) & \textit{// a triple of
%-%    an int, string, and character} \\
%-%*(\textbf{int}[],\textbf{int} f()) & \textit{// a pair of an integer
%-%    array, function}
%-%\end{tabular}
%-%}

\begin{verbatim}
*(int,int)         // the type of a pair of integers
*(int,string,char) // a triple of an int, string, and character
*(int[],int f())   // a pair of an integer array, function
\end{verbatim}

Tuples are ordered n-tuples of their respective component types.  Tuples
are particularly useful for returning multiple values or for packaging
up multiple values within a union, especially since Popcorn requires
that you name your struct types (i.e., does not support anonymous
structs.)   If e is an expression of type type, then the components of e
may be accessed by writing ``e.1'', ``e.2'', ``e.3'', etc.  New tuples
may be created by writing ``\^{} *($<$exp$>$,...,$<$exp$>$)'' or ``new
*($<$exp$>$,...,$<$exp$>$)''.

Currently, tuples are mutable but we might change this or might allow
additional qualifiers to declare whether a field is mutable or not.  We
expect to add tuple pattern support in the near future.

The default initial value for variables of tuple type is a tuple of the
default initial values of the respective component types.

\subsubsection{Function Type Expressions}

%-%\texttt{
%-%\begin{tabular}{ll}
%-%\textbf{int} f(\textbf{short},\textbf{short})
%-%& \textit{// a function that takes two shorts and returns an int} \\
%-%\textbf{int} (\textbf{short},\textbf{short})
%-%& \textit{// also a function that takes two shorts and returns an int} \\
%-%\textbf{int} (\textbf{short} x,\textbf{short} y) 
%-%& \textit{// ditto -- you can also name arguments} \\
%-%\textbf{int}[] (\textbf{char}) 
%-%& \textit{// takes a char and returns an integer array} \\
%-%\textbf{int} length<a>(<a>list)
%-%& \textit{// for all types a, takes an <a>list and} \\
%-%& \textit{// returns an int} \\
%-%\textbf{int} <a>(<a>list x)
%-%& \textit{// same as previous} \\
%-%\textbf{int} <a>(<a>list)
%-%& \textit{// same as previous} \\
%-%b <a,b>(b f(a,b), <a>list, b accum) 
%-%& \textit{// takes three arguments, the first is a function from a} \\
%-%& \textit{// and b to b, the second is an <a>list, and the third}  \\
%-%& \textit{// is a b.  The entire function returns a b value.} \\
%-%\textbf{int} (\textbf{int})[]
%-%& \textit{// an array of functions from int to int} \\
%-%<\textbf{int} (\textbf{int})>\textbf{array}
%-%& \textit{// an array of functions from int to int}
%-%\end{tabular}
%-%}

\begin{verbatim}
int f(short,short) // a function that takes two shorts and returns an int
int (short,short)  // also a function that takes two shorts and returns an int
int (short x,short y)  // ditto -- you can also name arguments
int[] (char)           // takes a char and returns an integer array
int length<a>(<a>list) // for all types a, takes an <a>list and 
                       // returns an int
int <a>(<a>list x)     // same as previous
int <a>(<a>list)       // same as previous
b  <a,b>(b f(a,b), <a>list, b accum) 
    // takes three arguments, the first is a function from a and b to
    // b, the second is an <a>list, and the third is a b.  The entire
    // function returns a b value.
int (int)[]            // an array of functions from int to int
<int (int)>array       // an array of functions from int to int
\end{verbatim}

Function type syntax is by far the most confusing because C
fundamentally gets this wrong.  However, porting C code with a different
notation for types is quite difficult so we have attempted to keep the C
syntax as much as possible.

For monomorphic, first-order functions, the syntax is essentially the
same as in C:  the result type comes first, followed by an [optional]
identifier, then a left parenthesis, followed by a comma-separated list
of argument types, and then a right parenthesis.   In addition, the
arguments can be given an optional identifier for a name.  Optional
identifiers for the function name and the argument names are ignored by
the type system, so ``int f(short x,short y)'' is equivalent to ``int
(short,short)''.

For polymorphic functions, the type parameters must be explicit.   As in
C++, these are enclosed by ``$<$'' and ``$>$'' and are comma separated
type variables.   So, for instance, ``a id$<$a$>$(a)'' is the type of
the polymorphic identity function and is equivalent to ``a (a)''.
Declaring a function and its type is detailed in Section [?].  There is
no default value for variables of function type -- that is, programmers
must explicitly supply such a value.

\subsubsection{The Exception Type}

%-%\texttt{\textbf{exn}}

\begin{verbatim}
exn
\end{verbatim}


Exception values are created with a new expression by applying an
exception constructor to a value of the appropriate type.  The resulting
value has type exn and may be used by a raise expression.  When a
try/handle expression is executed, if the body of the try block raises
an exception, then the handle is given an exn value.  The handle
expression typically performs a switch on the exn value to determine
which exception was raised and to extract any value(s) placed in the exn
data structure.  In this respect, exn values are much like ML's exn
type.  New exception constructors are declared using the exception
declaration (see Section [?]).

There is no default value for variables of exn type.

\subsubsection{Type Constructor Expressions}

%-%\texttt{
%-%\begin{tabular}{ll}
%-%bar
%-%& \textit{// the bar type takes no arguments} \\
%-%<\textbf{int}>list
%-%& \textit{// the list type takes one argument -- int in this case} \\
%-%<\textbf{int},\textbf{string}>dict
%-%& \textit{// the dict type takes two arguments: int \& string here} \\
%-%<\textbf{int}[]>list
%-%& \textit{// a list of integer arrays} \\
%-%<\textbf{<int}>\textbf{array}>list
%-%& \textit{// a list of integer arrays}
%-%\end{tabular}
%-%}

\begin{verbatim}
bar                // the bar type takes no arguments
<int>list          // the list type takes one argument -- int in this case
<int,string>dict   // the dict type takes two arguments: int & string here
<int[]>list        // a list of integer arrays
<<int>array>list   // a list of integer arrays
\end{verbatim}

Type constructors are, as in ML, applied to type arguments to produce a
type expression.  Also as in ML, the application occurs on the left.  As
in C++, the arguments to the constructor are placed in ``$<$'' and
``$>$'' brackets and are comma-separated.

Type constructors are created either through a struct declaration (see
Section [?]), a union declaration (see Section [?]), or an abstype
declaration (see Section[?]).  In the examples above, let us assume that
we have declared the following struct declarations:

%-%\texttt{
%-%\begin{tabular}{l}
%-%\textbf{struct} bar
%-%    \{\textbf{int} x; \textbf{int} y; \textbf{int} z; \} \\
%-%?\textbf{struct} <a>list \{ a hd; <a>list tl; \} \\
%-%\textbf{struct} <a,b>dict 
%-%    \{ \textbf{int} comp(a,a); <*(a,b)>list contents; \}
%-%\end{tabular}
%-%}

\begin{verbatim}
struct bar {int x; int y; int z; }
?struct <a>list { a hd; <a>list tl; }
struct <a,b>dict { int comp(a,a); <*(a,b)>list contents; }
\end{verbatim}

Then bar is declared as a type constructor with no arguments, list is a
type constructor with one argument, and dict is a type constructor with
two arguments.   Using the type expression ``$<$int$>$list'' is similar to
writing the following monomorphic struct declaration and using it
instead:

%-%\texttt{
%-%?struct int{\_}list \{ \textbf{int} hd; int{\_}list tl; \}
%-%}

\begin{verbatim}
?struct int_list { int hd; int_list tl; }
\end{verbatim}

However, the type system treats ``$<$int$>$list'' and ``int{\_}list'' as
different types.

The default initial value of a ?struct type is null (similar to Java
objects).  The default initial value of a struct type is the value of
that type with fields containing the default initial values of the field
types of the struct declaration.  If any field type has no default
initial value, then neither does the struct type.  The default initial
value of a union type depends entirely on the first variant of the type
declaration.  If the variant has type void, the default value of the
untion type is a value with this variant.  Else the default value of the
union type is a value with this variant and its corresponding value
being the default value of the variant's type.  If this type has no
default value, then the union type has no default value.

Note: Default values may go away in the future.


\subsection{Declarations\label{declarations}}

\subsubsection{Top-Level Variable Declarations}

%-%\texttt{
%-%<top-var-decl> ::= [<scope>]<type-exp> <qid><type-modifiers>[= <const-exp>];
%-%}
%-%
%-%\texttt{
%-%\begin{tabular}{ll}
%-%\textbf{static int} x;
%-%& \textit{// a top-level, private variable x of type int} \\
%-%\textbf{int} x = 1;
%-%& \textit{// a top-level, public variable x of type int, initial value 1} \\
%-%\textbf{int} x,y=3,z;
%-%& \textit{// x, y, and z all public of type int, y has initial value 3} \\
%-%\textbf{static int}[] a;
%-%& \textit{// a is a private array of integers} \\
%-%\textbf{static int} a[];
%-%& \textit{// same as previous} \\
%-%\textbf{extern int }x;
%-%& \textit{// x is an externally-defined integer variable} \\
%-%\textbf{extern string} z;
%-%& \textit{// z is an externally-defined string variable}
%-%\end{tabular}
%-%}

\begin{verbatim}
<top-var-decl> ::= [<scope>]<type-exp> <qid><type-modifiers>[= <const-exp>];

static int x; // a top-level, private variable x of type int
int x = 1;    // a top-level, public variable x of type int, initial value 1

int x,y=3,z;    // x, y, and z all public of type int, y has initial value 3
static int[] a; // a is a private array of integers
static int a[]; // same as previous

extern int x;     // x is an externally-defined integer variable
extern string z;  // z is an externally-defined string variable
\end{verbatim}

Top-level declarations for variables are prefixed with an optional scope
modifier (either ``extern'' or ``static''), followed by a type and then
the variable named, followed by an optional initial value.  Multiple
variable definitions may be collapsed if they have the same scope and
type.  Initial values for top-level expressions must be ``constant''
expressions and may not include function calls.  Constant expressions
are detailed in Section [?], but roughly speaking consist of integer
constants, string and array constants, new expressions (where the
arguments are constants), null, and a few other expressions.  Top-level
variables are not allowed to be polymorphic, though they may contain
values that refer to polymorphic functions.

To support C-style declarations of variables of array type or functional
type, it is possible to move part of the type expression onto the
variable being declared.  For instance, we may write ``int x[]'' to
declare x to be an array of integers.

The static qualifier defines a variable whose scope is the current
compilation unit (usually a file).  That is, the variable is only
accessible to those functions defined in the same compilation unit.

The extern qualifier claims that such a variable is defined by some
other compilation unit with the associated type.  Obviously, only one of
static or extern may apply to a given definition.

If no initializer is provided for the variable, then the compiler
implicitly inserts an initializer.  Not all types admit a default
initial value and so programmers must provide them.  Refer to Section
[?] which describes types to see which types admit default initial
values and if so, what those values are.

In the future, we expect additional qualifiers including ``const''
(immutable variables).  The restriction that all variables must be
initialized is a painful one, especially with top-level variables of
abstract type and we hope to lift this restriction some day.  However,
the TAL module system only supports primitive data values (much like a
standard assembler) and does not support function calls for initializing
variables, hence the restriction.  In the meantime, it is relatively
easy to work around the restriction.  For example, if you wish to define
a variable x of abstract type FOO but don't have an initial value of
type FOO, then you can instead define x to have type $<$FOO$>$Opt where
Opt is a struct defined as:

%-%\texttt{
%-%?\textbf{struct} <a>Opt \{ a v; \}
%-%}

\begin{verbatim}
struct <a>Opt { a v; }
\end{verbatim}

(For ML users, Opt is similar to the option datatype.)  Initially, x
will have value null.  The value may then be later initialized (e.g., in
main()) as follows:  If e is the expression that computes the initial
FOO value, we assign:   ``x = \^{} Opt(e)''  (i.e., x gets new Opt(e)).  To
extract or modify the value from x, it is necessary to write ``x.v''
instead of just ``x''.   Note that the use of ``x.v'' causes an implicit
check for null to be inserted by the compiler which is necessary to
ensure that the variable is not used before it is initialized.  Finally,
note that the Opt constructor is defined as above in $<$core.h$>$.

Top-level variables may not currently have a function type.  This
(somewhat rare) inconvenience can be circumvented with a level of
indirection such as with the Opt type.

\subsubsection{Local Variable Declarations}

%-%\texttt{
%-%<var-decl> ::= <type-exp> <id>[<type-modifiers>][= <exp>];
%-%}

\begin{verbatim}
<var-decl> ::= <type-exp> <id>[<type-modifiers>][= <exp>];
\end{verbatim}

Similar to top-level declarations except that there are no scope
qualifiers.  Unlike Java, we require all variables to be initialized (or
else insert an implicit initialization to the default value for the
variable's type.)  We expect to lift this restriction someday, by
performing a simple flow-insensitive dataflow analysis (a la JVML) to
ensure that variables are initialized before they are used.

\subsubsection{Struct Declarations}

%-%\texttt{
%-%\begin{tabular}{lrl}
%-%<struct-decl>   & ::= & [<scope>] [?] \textbf{struct}
%-%                        [<<type-var>,...,<type-var>>]
%-%                        \{ <struct-field>;+ \} \\
%-%<struct-field>  & ::= & [\textbf{const}] <type-exp> <id>
%-%                        <type-modifiers>
%-%\end{tabular}
%-%}
%-%
%-%\texttt{
%-%\begin{tabular}{ll}
%-%\textbf{struct} point \{\textbf{int} x; \textbf{int} y; \}
%-%& \textit{// a record of two integer fields, x \& y} \\
%-%\textbf{struct} point \{\textbf{const} \textbf{int} x; \textbf{const}
%-%    \textbf{int} y; \}
%-%& \textit{// same but x and y are immutable} \\
%-%?\textbf{struct} point \{\textbf{int} x; \textbf{int} y; \}
%-%& \textit{// same as above but point values can be null} \\
%-%?\textbf{struct} pointlist \{point p; pointlist next; \}
%-%& \textit{// lists of points} \\
%-%\textbf{static} \textbf{struct} point \{\textbf{int} x,y;\}
%-%& \textit{// the type declaration is private} \\
%-%\textbf{abstract} \textbf{struct} point \{\textbf{int} x,y; \}
%-%& \textit{// the type declaration is exported abstractly} \\
%-%?\textbf{struct} <a>list \{a hd; <a>list tl; \}
%-%& \textit{// polymorphic lists} \\
%-%?\textbf{struct} <a,b,c>triple \{a x; b y; c z; \}
%-%& \textit{// polymorphic triple} \\
%-%\textbf{extern} \textbf{struct} point \{\textbf{int} x; \textbf{int}
%-%    y;\}
%-%& \textit{// point is defined publicly elsewhere} \\
%-%\textbf{extern} point;
%-%& \textit{// point is defined (perhaps abstractly) elsewhere} \\
%-%\textbf{extern} ?point;
%-%& \textit{// point is an ? struct defined elsewhere}
%-%\end{tabular}
%-%}

\begin{verbatim}
<struct-decl> ::= [<scope>] [?] struct [<<type-var>,...,<type-var>>] { <struct-field>;+ }
<struct-field>::= [const] <type-exp> <id> <type-modifiers>

struct point {int x; int y; }  // a record of two integer fields, x & y
struct point {const int x; const int y; } // same but x and y are immutable 
?struct point {int x; int y; } // same as above but point values can be null
?struct pointlist {point p; pointlist next; } // lists of points
static struct point {int x,y;} // the type declaration is private
abstract struct point {int x,y; } // the type declaration is exported abstractly
?struct <a>list {a hd; <a>list tl; } // polymorphic lists
?struct <a,b,c>triple {a x; b y; c z; } // polymorphic triple

extern struct point {int x; int y;} // point is defined publicly elsewhere
extern point;                       // point is defined (perhaps abstractly) elsewhere
extern ?point;                      // point is an ? struct defined elsewhere
\end{verbatim}

Struct declarations are essentially type declarations for ``pointers to
records of values''.  If no question mark appears before the struct
declaration, then values of that type are always valid pointers to
records (i.e., never null).  If a question mark appears before the type
(so-called option-structs), then the pointer may be null.  When
dereferencing an option-struct value, the compiler inserts implicit
checks for null and raises the NullPointerException exception if the
value is found to be null.  Hence, before setting or reading a field in
a struct, a programmer should always check to see whether the value is
null.

Before the struct declaration, one can write an optional scope
qualifier.  Lack of a qualifier indicates that the type declaration is
to be fully exported to the outside world.  The qualifier static
indicates that the type declaration is to be fully hidden to the outside
world.  The qualifier abstract indicates that the name of the type
constructor, the number of type arguments it takes, and whether it is an
option-struct  is exported to the outside world, but the definition is
not.  The extern qualifier is used when some other compilation unit
defines the struct (and exports it at least in part).  There is
currently no way to export a struct without revealing whether or not it
is an option-struct.

The fields of the struct are given types and names as identifiers and
may be optionally prefixed with ``const'' to indicate that the field is
immutable.  Fields are accessed using the ``dot'' notation.  For
example, to access the x component of a point  p we would write ``p.x''.
To create a struct value, we write ``new
$<$struct-name$>$($<$exp1$>$,...,$<$expn$>$)'' or ``\^{}
$<$struct-name$>$($<$exp1$>$,...,$<$expn$>$)'' where the expression
arguments are used as the initial values of the fields.  Thus, it is
impossible to create a struct value with uninitialized fields.

Alternatively, one may create a struct value by writing ``new
$<$struct-name$>$\{$<$field1$>$ =
$<$exp1$>$,...,$<$fieldn$>$=$<$expn$>$\}'' or more commonly, ``\^{}
$<$struct-name$>$\{$<$field1$>$=$<$exp1$>$,...,$<$fieldn$>$=$<$expn$>$\}''.
Unlike the notation above, the expressions need not be listed in the
order that the fields are declared.

Structs may be recursive as in the pointlist and list examples above.
Thus, the scope of a struct name includes its definition.  In fact, the
scope of a struct name is an entire compilation unit (except where
shadowed by a type variable), so mutually recursive struct definitions
can be made without the acrobatics required by languages such as C.

Structs may also be polymorphic in any number of type arguments.  For
parsing reasons, we placed the type parameters to the left of the
constructor instead of the right.  This allows us to easily determine
type expressions from polymorphic function declarations.  In the list
example above, a value of type $<$int$>$list is either null, or a pointer to
a struct that contains an int in its hd field, and an $<$int$>$list in its
tail field.  Any type but void may be used to instantiate the type
variables.

There is no need (and in fact, no way) to explicitly instantiate the
type variables of a polymorphic struct.  Rather, the type checker
automatically determines the instantiation through unification in a
style similar to ML type inference.  Similarly, when one writes
``null'', it can represent any option-struct -- there is no need to
explicitly say which type.  We expect that sometime in the future, we
will provide syntax for explicitly instantiating a polymorphic type
constructor and for explicitly writing at which type null is meant to
be.  As a simple example, the following polymoprhic function calculates
the length of a list:

%-%\texttt{
%-%\textbf{int} list{\_}length<a>(<a>list x) \{ \newline
%-%~~\textbf{int} len = 0; \newline
%-%\newline
%-%~~\textbf{while} (x != \textbf{null}) \{ \newline
%-%~~~~len++; \newline
%-%~~~~x = x.tl; \textit{// implicit check for null here} \newline
%-%~~~~\} \newline
%-%~~\textbf{return}(len) \newline
%-%\}
%-%}

\begin{verbatim}
int list_length<a>(<a>list x) {
  int len = 0;

  while (x != null) {
    len++;
    x = x.tl;  // implicit check for null here
  }
  return(len)
}
\end{verbatim}

Notice that the code is able to calculate the length regardless of the
type of the components in the list.  Notice also that we explicitly
check for null before dereferencing the list.  Currently, the popcorn
compiler inserts an implicit check for null at the point where ``x.tl''
is dereferenced, but as the optimizations in the compiler are improved,
we expect this overhead to disappear.  As another example, the following
code computes and returns a list of the first n integers:

%-%\texttt{
%-%<\textbf{int}>list f(\textbf{int} n) \{ \newline
%-%~~<\textbf{int}>list r; \textit{// implicitly initialized to null} \newline
%-%\newline
%-%~~\textbf{while} (n != 0) \{ \newline
%-%~~~~r = \^{} list(n,r); \textit{// new list struct created here} \newline
%-%~~~~r = \^{} list\{tl=r,hd=n\}; \textit{// equivalent to previous line}
%-%    \newline
%-%~~~~n--; \newline
%-%~~\} \newline
%-%~~\textbf{return}(r); \newline
%-%\}
%-%}

\begin{verbatim}
<int>list f(int n) {
  <int>list r;  // implicitly initialized to null

  while (n != 0) {
    r = ^list(n,r);  // new list struct created here
    r = ^list{tl=r,hd=n};  // equivalent to previous line
    n--;
  }
  return(r);
}
\end{verbatim}

Notice that the variable r is implicitly initialized to the value null.
In practice, implicit initialization like this is not a good idea, and
one should write ``$<$int$>$list r = null'' to make the code more clear.
Notice also that in the body of the while loop, a new list node is
created using ``\^{} list(n,r)''.  One could also write ``new list(n,r)''
to make the code appear a bit more like Java (at the price of more
keystrokes).  Alternatively, we may write ``\^{} list\{tl=r,hd=n\}'' or
``new \{hd=n,tl=r\}'' as in ML.  Using the explicit field names means
that we don't have to worry about getting the order of the fields right.

Type equivalence on structs is not done structurally as in Standard ML
or in C.  Rather, struct equivalence is by name.  Thus, if one defines:

%-%\texttt{
%-%\textbf{struct} point1 \{\textbf{int} x; \textbf{int} y;\} \newline
%-%\textbf{struct} point2 \{\textbf{int} x; \textbf{int} y;\} \newline
%-%\newline
%-%\textbf{int} f(point1 p) \newline
%-%~~\{ \textbf{return}(p.x + p.y) \}
%-%}

\begin{verbatim}
struct point1 {int x; int y;}
struct point2 {int x; int y;}

int f(point1 p) 
{ return(p.x + p.y) }
\end{verbatim}

Then one may pass a point1 value to f, but not a point2 value.  The
decision to use name equivalence instead of structural equivalence makes
supporting recursive struct definitions easier at the TAL level and it
makes it easier to build a type-preserving compiler, because we need not
expand type definitions eagerly.   However, it interferes with the
addition of several desirable features including sub-typing so this
decision will perhaps be re-visited in the future.

There is no current support for ``flattening'' one struct into another,
nor for creating arrays of flattened structs.  Rather, as in Java,  we
are forced to manipulate pointers (or references) to the record.  Though
TAL supports flattened records and structs, we have not yet found a
simple, elegant way to export this functionality to a source language
like Popcorn but hope to some day.

\subsubsection{Union Declarations}

%-%\texttt{
%-%\begin{tabular}{lrl}
%-%<union-decl> & ::= & [<scope>]
%-%                     \textbf{union}[<<type-var>,...,<type-var>>] \{
%-%                     <type-exp> <id> <type-modifiers>; ... <type-exp>
%-%                     <id> <type-modifiers>; \}
%-%}
%-%
%-%\texttt{
%-%\textbf{union} <a>list \{ \textbf{void} Null; *(a,<a>list) Cons; \}
%-%    \newline
%-%\textit{
%-%~~~~// the list declaration above is similar to the ?struct \newline
%-%~~~~// declaration given in the previous section, and approximates
%-%    \newline
%-%~~~~// the ML datatype:  datatype 'a list = Nil | Cons of 'a*'a list
%-%    \newline
%-%~~~~// however, the union approach requires an extra level of
%-%    indirection \newline
%-%~~~~// for each Cons cell. \newline
%-%}
%-%\newline
%-%\textbf{union} exp \{ \textbf{string} Var; *(\textbf{string},exp)
%-%Lambda; *(exp,exp) App; \} \newline
%-%~~~~\textit{// a recursive tagged union representing lambda expressions}
%-%}

\begin{verbatim}
<union-decl> ::= [<scope>] union[<<type-var>,...,<type-var>>] {
                <type-exp> <id> <type-modifiers>; ... <type-exp> <id> <type-modifiers>; }

union <a>list { void Null; *(a,<a>list) Cons; }
  // the list declaration above is similar to the ?struct 
  // declaration given in the previous section, and approximates
  // the ML datatype:  datatype 'a list = Nil | Cons of 'a*'a list
  // however, the union approach requires an extra level of indirection
  // for each Cons cell.

union exp { string Var; *(string,exp) Lambda; *(exp,exp) App; }
  // a recursive tagged union representing lambda expressions
\end{verbatim}

Unions in Popcorn are more like ML datatypes than like C unions.   That
is, union values are fundamentally disjoint or tagged sums.  The tag is
necessary so that we can dynamically determine the actual type carried
by the union value.   As with structs (and ML-style datatypes), unions
can be recursive and polymorphic, and their visibility can be controlled
via public, static, abstract, and extern qualifiers.  (See Section [?].)

The field names of a union are used as the datatype constructors.  To
create a value of a union type we write
``\^{} $<$union-name$>$.$<$field-name$>$($<$arg1$>$,...,$<$argn$>$)'' (or
use the keyword ``new'' instead of ``\^{}''.)  For example, to create an
exp value representing the variable x, we would write
``\^{} exp.Var("x")''.

Union fields are not mutable.

When we have a value x of some union type, then we can use a switch
expression to determine what its tag and underlying value is, similar to
the case-expression and pattern matching constructs of SML.  For
instance, the following function prints out exp values:

%-%\texttt{
%-%\textbf{void} print{\_}exp(exp e) \{ \newline
%-%~~\textbf{switch} (e) \{ \newline
%-%~~~~\textbf{case} Var(x):  \textbf{printf}("\%s",x); \newline
%-%~~~~\textbf{case} Lam(p):  \textbf{printf}("(lambda \%s ",p.1); \newline
%-%~~~~~~~~~~~~~~~~~~print{\_}exp(p.2); \newline
%-%~~~~~~~~~~~~~~~~~~\textbf{printf}(")"); \newline
%-%~~~~\textbf{case} App(p):  \textbf{printf}("("); print{\_}exp(p.1);
%-%    \textbf{printf}(" "); \newline
%-%~~~~~~~~~~~~~~~~~~print{\_}exp(p.2); \textbf{printf}(")"); \newline
%-%~~\} \newline
%-%\}
%-%}

\begin{verbatim}
void print_exp(exp e) {
  switch (e) {
    case Var(x):  printf("%s",x);
    case Lam(p):  printf("(lambda %s ",p.1);
                  print_exp(p.2);
                  printf(")");
    case App(p):  printf("("); print_exp(p.1); printf(" ");
                  print_exp(p.2); printf(")");
  }
}
\end{verbatim}

A switch on a union type requires cases for each of the fields (or else
a default: case).  Each field's case also defines a local variable (or
more generally a pattern) that has the type of the field which may be
used locally within the switch-case-clause.  For instance, in the
example above, the Var field has type string, so the x variable
associated with the Var case has type string within the block of code
corresponding to that case.  The Lam field has type *(string,exp) (a
tuple of a string and expression) so p is assigned this type within the
Lam case.  Finally, the App field has type *(exp,exp) so p is assigned
this type within the App case.  Cases for void variants do not define
local variables.  Notice that unlike C, switch cases do \textit{not}
implicitly fall through to the next case, primarily because the scope of
variables bound within the cases does not extend across cases.
Correspondingly, break does \textit{not} cause control to be transferred
out of the switch but rather out of the nearest enclosing loop.  In the
future, we may change this by forcing programmers to explicitly write
``break'' to make it easier to port C code.

It is also possible to deconstruct a union value by simply using the
``dot'' notation.  For instance, we can write ``e.Var'' and this is
equivalent to writing ``switch (e) \{ case Var(x): return(x); default:
raise UnionVariantException;\}''.

Wildcards (``{\_}'') and tuple-patterns (``*(x1,...,xn)'') may also be
used within a switch to provide a bit more pattern matching a la ML.  In
particular, the switch expression above may be rewritten as follows:

%-%\texttt{
%-%\textbf{switch} (e) \{ \newline
%-%~~\textbf{case} Var(x):  printf("\%s",x); \newline
%-%~~\textbf{case} Lam*(var,body):  \textit{// var and body form a
%-%  tuple-pattern} \newline
%-%~~~~~~~~~~~~~~~~printf("(lambda \%s ",var); \newline
%-%~~~~~~~~~~~~~~~~print{\_}exp(body); \newline
%-%~~~~~~~~~~~~~~~~printf(")"); \newline
%-%~~\textbf{case} App*(fn,arg):    \textit{// fn and arg form a
%-%  tuple-pattern} \newline
%-%~~~~~~~~~~~~~~~~printf("("); print{\_}exp(fn); printf(" "); \newline
%-%~~~~~~~~~~~~~~~~print{\_}exp(arg); printf(")"); \newline
%-%\}
%-%}

\begin{verbatim}
  switch (e) {
    case Var(x):  printf("%s",x);
    case Lam*(var,body):  // var and body form a tuple-pattern
                  printf("(lambda %s ",var);
                  print_exp(body);
                  printf(")");
    case App*(fn,arg):    // fn and arg form a tuple-pattern
                  printf("("); print_exp(fn); printf(" ");
                  print_exp(arg); printf(")");
  }
\end{verbatim}


\subsubsection{Abstype Declarations}

%-%\texttt{
%-%\begin{tabular}{lrl}
%-%<abstype-decl> & ::= & [<scope>] \textbf{abstype} 
%-%                       [<<type-var>,...<type-var>>]id
%-%                       [[<type-var>,...,<type-var>]] = <type-exp>;
%-%\end{tabular}
%-%}

\begin{verbatim}
<abstype-decl> ::= [<scope>] abstype 
                   [<<type-var>,...<type-var>>]id
                   [[<type-var>,...,<type-var>]] = <type-exp>;
\end{verbatim}

Abstypes are a form of first-class abstract data types, similar in many
respects to a very primitive form of object type. They are particularly
useful when one wants to manipulate heterogeneous data structures. Like
a struct or union, an abstype can be polymorphic.  Unlike structs or
unions, an abstype can \textit{abstract} or \textit{hide} certain types
as well.  The typical use of an abstype is when we want to export some
but not all information about a type.  For instance, when representing
objects, we may want to hide the types of the instance variables but
expose the types of the methods.  Furthermore, the methods should take
the instance variables as extra arguments.  As another example, a
\textit{closure} may be represented by an abstract environment and a
function which, when given the environment and an argument, produces a
result.

Like structs and unions, abstype values are created by using the ``new''
or ``\^{} '' construct.  To manipulate an abstype value, we must use the
``with'' construct to open up the abstracted type in some scope.

As a simple example, suppose we have two different representations for
2-dimensional points, polar and caretesian, with appropriate operations
defined on them:

%-%\texttt{
%-%\textbf{extern struct} polar \{\textbf{int} xcoord; \textbf{int}
%-%    ycoord;\} \newline
%-%\textbf{extern} polar add{\_}polar(polar x, polar y); \newline
%-%\textbf{extern} polar sub{\_}polar(polar x, polar y); \newline
%-%\textbf{extern} polar mul{\_}polar(polar x, polar y); \newline
%-%\newline
%-%\textbf{extern} \textbf{struct} cartesian \{\textbf{int} mag;
%-%    \textbf{int} angle;\} \newline
%-%\textbf{extern }cartesian add{\_}cart(cartesian x, cartesian y); \newline
%-%\textbf{extern }cartesian sub{\_}cart(cartesian x, cartesian y); \newline
%-%\textbf{extern }cartesian mul{\_}cart(cartesian x, cartesian y);
%-%}

\begin{verbatim}
extern struct polar {int xcoord; int ycoord;}
extern polar add_polar(polar x, polar y);
extern polar sub_polar(polar x, polar y);
extern polar mul_polar(polar x, polar y);

extern struct cartesian {int mag; int angle;}
extern cartesian add_cart(cartesian x, cartesian y);
extern cartesian sub_cart(cartesian x, cartesian y);
extern cartesian mul_cart(cartesian x, cartesian y);
\end{verbatim}

Unfortunately, Popcorn does not allow one to mix polar and cartesian
objects directly.  For example, it's impossible to have a list that
mixes polar and cartesian points.  Now one could define a point union
with a field for polars and a field for cartesians, and then manipulate
lists of these unions.  But what if we want to add a third kind of
point?  Abstypes allow us to abstract the particular representation of a
type (in this case, whether a point is polar or cartesian) and package
up the operations on values of those types.  For example, we might
define a generic point object as follows:

%-%\texttt{
%-%\begin{tabbing}
%-%\textbf{struct }<a>point{\_}rep \{ \=a data; \\
%-%                                   \>a add{\_}point(a,a); \\
%-%                                   \>a sub{\_}point(a,a); \\
%-%                                   \>a mul{\_}point(a,a); \}; \\
%-%\\
%-%\textbf{abstype }point[p] = <p>point{\_}rep;
%-%\end{tabbing}
%-%}

\begin{verbatim}
struct <a>point_rep { a data; 
                      a add_point(a,a);
                      a sub_point(a,a);
                      a mul_point(a,a); };

abstype point[p] = <p>point_rep;
\end{verbatim}

This definition defines a new type point that hides or abstracts the
representation of the field data (p) allowing us to mix different point
representations.  For example, we might define:

%-%\texttt{
%-%\begin{tabbing}
%-%point polar{\_}point(int mag, int angle) \{
%-%~~<polar>point{\_}rep pol = \^{} point{\_}rep(\=\^{} polar(mag, angle), \\
%-%                                             \>add{\_}polar,
%-%                                               sub{\_}polar,
%-%                                               mul{\_}polar); \\
%-%~~\textbf{return} \^{} point(pol); \\
%-%\}
%-%\end{tabbing}
%-%}
%-%
%-%\texttt{
%-%\begin{tabbing}
%-%point cartesian{\_}point(int x, int y) \{
%-%~~<cartesian>point{\_}rep car = \^{} point{\_}rep(\=\^{} cartesian(x, y),
%-%  \\
%-%                                                 \>add{\_}cart,
%-%                                                   sub{\_}cart,
%-%                                                   mul{\_}cart); \\
%-%~~\textbf{return} \^{} point(car); \\
%-%\}
%-%\end{tabbing}
%-%}

\begin{verbatim}
point polar_point(int mag, int angle) {
  <polar>point_rep pol = ^point_rep(^polar(mag, angle),
                                    add_polar, sub_polar, mul_polar);
  return ^point(pol); 
}

point cartesian_point(int x, int y) {
  <cartesian>point_rep car = ^point_rep(^cartesian(x, y),
                                        add_cart, sub_cart, mul_cart);
  return ^point(car);
}
\end{verbatim}

Notice that both function definitions return point values and that the
return type makes no mention of whether the representation of the point
is polar or cartesian.  Indeed, at the point where we create a point
(\^{} point(pol) or \^{} point(car)), we've abstracted the representation.
This is a lot like casting an object of a particular class in Java to
one of the interface types that the class implements -- you lose the
specific information about what kind of object it is and must manipulate
it through its abstract interface.

With these two definitions for creating points that are either polar or
cartesian, we can define, for instance, a list that mixes both kinds of
points:

%-%\texttt{
%-%<point>list points = \^{} list(polar{\_}point(10,15), \newline
%-%~~~~~~~~~~~~~~~~~~~~~~~~~~~\^{} list(cartesian{\_}point(0,0), \newline
%-%~~~~~~~~~~~~~~~~~~~~~~~~~~~~~~~~~\^{} list(polar{\_}point(3,3),null)));
%-%}

\begin{verbatim}
<point>list points = ^list(polar_point(10,15),
                           ^list(cartesian_point(0,0),
                                 ^list(polar_point(3,3),null)));
\end{verbatim}

We can then write a function to manipulate the list through the exposed
interface.  A simple example follows:

%-%\texttt{
%-%<a>point{\_}rep double{\_}point{\_}rep<a>(<a>point{\_}rep pr) \{ \newline
%-%~~a new{\_}data = pr.add{\_}point(pr.data,pr.data); \newline
%-%~~\textbf{return}
%-%  \^{} point{\_}rep(new{\_}data,pr.add{\_}point,pr.sub{\_}point,pr.mul{\_}point);
%-%  \newline
%-%\}
%-%}
%-%
%-%\texttt{
%-%point double{\_}point(point x) \{ \newline
%-%~~\textbf{with} pr[p] = x \{ \newline
%-%~~~~<p>point{\_}rep new{\_}pr = double{\_}point{\_}rep(pr); \newline
%-%~~~~\textbf{return} \^{} point(new{\_}pr); \newline
%-%~~\} \newline
%-%\}
%-%}
%-%
%-%\texttt{
%-%List::map(double{\_}point, points);
%-%}

\begin{verbatim}
<a>point_rep double_point_rep<a>(<a>point_rep pr) {
   a new_data = pr.add_point(pr.data,pr.data);
   return ^point_rep(new_data,pr.add_point,pr.sub_point,pr.mul_point);
}

point double_point(point x) {
   with pr[p] = x {
     <p>point_rep new_pr = double_point_rep(pr);
     return ^point(new_pr);
   }
}

List::map(double_point, points);
\end{verbatim}


In this example code, we first define a polymorphic function which, when
given a point{\_}rep where the data has type a, we return a new point
representation where the data has the same type.  We do so by simply
adding the old point representation data to itself and packaging this up
with the operations.  But this function manipulates $<$a$>$point{\_}rep
values, not points.  The double{\_}point function does what we need to
take a point with any representation and apply the
double{\_}point{\_}rep function to that representation.

The body of double{\_}point uses a \textit{with} statement to ``open
up'' the abstract point x.  Within the scope of the \textit{with}
statement, the variable pr is bound to the point representation and the
type variable p is bound to the type of the point representation's data.
That is, pr has type $<$p$>$point{\_}rep (for some type named by p) and
can only be used within the scope of the \textit{with} body.  Within the
body, we call the double{\_}point{\_}rep function on pr.  Given any type
p, double{\_}point{\_}rep will take a $<$p$>$point{\_}rep and produce a
$<$p$>$point{\_}rep, thus we get back a $<$p$>$point{\_}rep for a
result.  We then abstract p again by placing the new $<$p$>$point{\_}rep
value new{\_}pr in a point.  Finally, we return the new point as the
result of the function.

The \textit{with} statement allows us to unpack or open up an abstract
data type within some scope.  All this really means is that it gives us
a way to \textit{name} the type for a limited amount of code and to get
to the underlying value.  Notice that if we open up two different
points, then the type-checker forces us to use different names for the
two points and thus, their respective point representation types cannot
be treated as the same:

%-%\texttt{
%-%point point1,point2; \newline
%-%\newline
%-%\textbf{with} pr1[p1] = point1 \{ \newline
%-%~~\textbf{with} pr2[p2] = point2 \{ \newline
%-%~~~~pr2.add{\_}point(p1.data,p2.data); \textit{// fails to type-check!}
%-%    \newline
%-%~~\} \newline
%-%\}
%-%}

\begin{verbatim}
point point1,point2;

with pr1[p1] = point1 {
  with pr2[p2] = point2 {
    pr2.add_point(p1.data,p2.data);  // fails to type-check!
  }
}
\end{verbatim}

In the above example, the attempt to add a point1's data and point2's
data fails to type-check.  The reason is that point1's representation
might be incompatible with point2's representation.  One might be
tempted to rewrite the code so that we replace p2 with p1:

%-%\texttt{
%-%\textbf{with} pr1[p1] = point1 \{ \newline
%-%~~\textbf{with} pr2[p1] = point2 \{ \textit{// fails to
%-%  type-check!} \newline
%-%~~~~pr2.add{\_}point(p1.data,p2.data);   \newline
%-%~~\} \newline
%-%\}
%-%}

\begin{verbatim}
with pr1[p1] = point1 {
  with pr2[p1] = point2 {     // fails to type-check!
    pr2.add_point(p1.data,p2.data);  
  }
}
\end{verbatim}

but the Popcorn type-checker will reject this.  In general, Popcorn
requires that you keep distinct type variables syntactically distinct
(i.e., it does not implicitly alpha-vary them for you.)  This avoids a
number of potential problems and keeps the language type-safe.

Another simple example of using abstypes is to define closures.  A
closure is a pair of some code (a function pointer) and an environment.
The code is intended to take the environment and an argument and produce
a result.  Different closures may want to have different types of
environments, even though they take and return arguments and results of
the same type.  Without abstypes, we'd have to assign such closures
different types.  But in functional languages like ML, no such
distinction is made in their type.  Fortunately, we can use abstypes to
hide or abstract the type of the environment in a closure.  Furthermore,
we can use Popcorn's polymorphism to build closures and operations on
them once and for all.  The following code defines an abstype for
closures and a polymorphic composition function which, when given a
closure from a$\mapsto$b and a closure from b$\mapsto$c, produces a
closure from a$\mapsto$c.  The code takes advantage of the ability to
define local, static functions.

%-%\texttt{
%-%\textbf{abstype} <a,b>closure[c] = *(b code(env c,arg a), c); \newline
%-%\newline
%-%<a,c>closure compose<a,b,c>(<a,b>closure f1, <b,c>closure f2) \{ \newline
%-%\newline
%-%~~~~~c code<a,b,c>(*(<a,b>closure,<b,c>closure) env, a arg) \{ \newline
%-%~~~~~~~\textbf{with} f1[d] = env.1 \{ \newline
%-%~~~~~~~~~\textbf{with} f2[e] = env.2 \{ \newline
%-%~~~~~~~~~~~b f1code(d,a) = f1.1; \newline
%-%~~~~~~~~~~~c f2code(e,b) = f2.2; \newline
%-%~~~~~~~~~~~d f1env = f1.2; \newline
%-%~~~~~~~~~~~e f2env = f2.2; \newline
%-%~~~~~~~~~~~\textbf{return} f2code(f2env,f1code(f1env,arg)); \newline
%-%~~~~~~~~~\} \newline
%-%~~~~~~~\} \newline
%-%~~~~~\}; \newline
%-%\newline
%-%~~~~~\textbf{return }\^{} closure(code,\^{} (f1,f2)); \newline
%-%\}
%-%}

\begin{verbatim}
abstype <a,b>closure[c] = *(b code(env c,arg a), c);

<a,c>closure compose<a,b,c>(<a,b>closure f1, <b,c>closure f2) {

     c code<a,b,c>(*(<a,b>closure,<b,c>closure) env, a arg) {
       with f1[d] = env.1 { 
        with f2[e] = env.2 {
          b f1code(d,a) = f1.1;
          c f2code(e,b) = f2.2;
          d f1env = f1.2;
          e f2env = f2.2;
          return f2code(f2env,f1code(f1env,arg));
        }
       }
     };

     return ^closure(code,^(f1,f2));
}
\end{verbatim}


\subsubsection{Exception Declarations}

%-%\texttt{
%-%\begin{tabular}{lrl}
%-%<exn-decl> & ::= & [<scope>] \textbf{exception} <qid>; | 
%-%                   [<scope>]~\textbf{exception}~<qid>(<type-exp>);
%-%\end{tabular}
%-%}

\begin{verbatim}
<exn-decl>  ::= [<scope>] exception <qid>; | 
                [<scope>] exception <qid>(<type-exp>);
\end{verbatim}

Exceptions may only be declared at the top-level.  As with identifiers
and other values, the exception's scope can be controlled by an optional
scope qualifier.  Exceptions can carry values of a specific type, in
which case the second syntactic form should be used.

\subsubsection{Function Declarations}

Very similar to C except that there are provisions for polymorphism.


\subsection{Expressions\label{expressions}}

Most expressions have the same syntax and (intended) semantics as in C
or Java.  We detail the various expression forms here for completeness
and give examples.  The grammar for the abstract syntax of expressions
is as follows:

%-%\texttt{
%-%\begin{tabular}{lrl}
%-%<exp> & ::= & <const-exp> | <qid> | <if-exp> | <assign-exp> | <inst-exp>
%-%              | <new-exp> | <dot-exp> | <subscript-exp> | <raise-exp> |
%-%              <seq-exp> | <cast-exp> | <fun-call> | <fun-exp> |
%-%              <sprintf-exp>
%-%\end{tabular}
%-%}

\begin{verbatim}
<exp> ::=  <const-exp> | <qid> | <if-exp> | <assign-exp> | <inst-exp> |
           <new-exp> | <dot-exp> | <subscript-exp> | <raise-exp> | 
           <seq-exp> | <cast-exp> | <fun-call> | <fun-exp> | <sprintf-exp>
\end{verbatim}


\subsubsection{Constant Expressions}

%-%\texttt{
%-%\begin{tabular}{lrl}
%-%<const-exp> & ::= & \textbf{true} | \textbf{false} | <number> | <char> |
%-%                    <string> | \textbf{null} | <fun-inst> |
%-%                    \^{} *(<const-exp>,...,<const-exp>) |
%-%                    \^{} <qid>[.<id>](<const-exp>,...,<const-exp>) |
%-%                    \^{} <qid>\{<id>=<const-exp>,...,<id>=<const-exp>\} |
%-%                    \{<const-exp>,...,<const-exp>\} | \{:<type-exp>\}
%-%\end{tabular}
%-%}

\begin{verbatim}
<const-exp> ::= true | false | <number> | <char> | <string> | null |
                <fun-inst> |  ^*(<const-exp>,...,<const-exp>) | 
                ^<qid>[.<id>](<const-exp>,...,<const-exp>) | 
                ^<qid>{<id>=<const-exp>,...,<id>=<const-exp>} |
                {<const-exp>,...,<const-exp>} | {:<type-exp>} 
\end{verbatim}

Constant expressions include boolean constants (true, false), integer
constants (e.g., 345, 0xfeedface), character constants (e.g., 'a', 'b',
'$\backslash$n', '$\backslash$000), string constants (e.g., ``foo'',
``bar$\backslash$n''), and null.  Constructed constant expressions
include tuples of constant expressions (e.g.,
*($<$const-exp$>$,...,$<$const-exp$>$)), new structs and unions of
constant expressions (e.g., list(3,null)), and array constant
expressions (e.g., \{3,4,5,6\}).   The key use of constant expressions
is to serve as initializers for top-level data structures.

In the future, we may allow things like arithmetic on constant
expressions and a few other forms.

\subsubsection{Variables}

%-%\texttt{
%-%\begin{tabular}{lrl}
%-%<qid> & ::= & [<id>::]<id> \\
%-%<id>  & ::= & [a-zA-Z][a-zA-Z0-9{\_}]*
%-%\end{tabular}
%-%}

\begin{verbatim}
<qid> ::= [<id>::]<id>
<id>  ::= [a-zA-Z][a-zA-Z0-9_]*
\end{verbatim}

Value variables are composed of a sequence of upper and lower-case
letters, digits, and underscore, except that variables may not begin
with an underscore.  Qualified identifiers are prefixed with an
identifier (usually with the first-letter in uppercase) followed by
``::'' (double-colon).   When used on the right-hand side of an
expression, they denote the value contained in the variable.  When used
as the left-hand-side of an assignment, they denote the location of the
variable (as in C).

\subsubsection{Conditional Expressions}

%-%\texttt{
%-%\begin{tabular}{lrl}
%-%<if-exp> & ::= & <exp> ? <exp> : <exp>
%-%\end{tabular}
%-%}

\begin{verbatim}
<if-exp> ::= <exp> ? <exp> : <exp>
\end{verbatim}

The first expression must have type bool and the other two expressions
must have the same type.  When evaluated, if the first-expression yield
true, then the second expression is evaluated and the result is returned
as the value of the entire expression.  Otherwise, when the
first-expression yields false, the third expression is evaluated.

\subsubsection{Assignment Expressions}

%-%\texttt{
%-%\begin{tabular}{lrl}
%-%<assign-exp> & ::= & <lhs-exp> = <exp> | <lhs-exp> <arithop>= <exp> |
%-%                     <lhs-exp>++ | <lhs-exp>-- | ++<lhs-exp> |
%-%                     --<lhs-exp> \\
%-%<lhs-exp>    & ::= & <qid> | <lhs-path> \\
%-%<lhs-path>   & ::= & <exp>.<id> | <exp>.<num> | <exp>[<exp>] |
%-%                     <lhs-path>.<id> | <lhs-path>[<exp>]</pre>
%-%\end{tabular}
%-%}

\begin{verbatim}
<assign-exp> ::= <lhs-exp> = <exp> | <lhs-exp> <arithop>= <exp> |
                 <lhs-exp>++ | <lhs-exp>-- | ++<lhs-exp> | --<lhs-exp>
<lhs-exp>    ::= <qid> | <lhs-path>
<lhs-path>   ::= <exp>.<id> | <exp>.<num> | <exp>[<exp>] | 
                 <lhs-path>.<id> | <lhs-path>[<exp>]
\end{verbatim}

The normal case of an assignment is of the form $<$id$>$ = $<$exp$>$ and simply
evaluates the given expression and assigns the resulting value to the
given variable.  The value of the assignment is the value placed into
the variable as in C.   In general, the left-hand-side can be a field of
a struct or tuple or some subscripted element of an array.  In addition,
Popcorn supports a number of the shortcut assignment operators provided
by C and Java.  For instance, the expressions ``x += 1'' and ``++x'' are
both equivalent to ``x = x + 1''.

\subsubsection{Function Calls}

%-%\texttt{
%-%\begin{tabular}{lrl}
%-%<fun-call> & ::= & <exp>(<exp>,...,<exp>)
%-%\end{tabular}
%-%}

\begin{verbatim}
<fun-call> ::= <exp>(<exp>,...,<exp>)
\end{verbatim}

The first expression must evaluate to a function.  If the function is
not polymorphic, then the argument expressions must have types equal to
the formal argument types of the function, and the type of the entire
expression is the result type of the function.  If the function is
polymorphic, then there must exist some instantiation of the function's
type variables such that the argument types match.   The same
instantiation is used to calculate the result type.  For example, in a
context where the function f has type a (*(a,b)) (a polymorphic function
accepting 2-tuples with any pair of types returning something of the
same type as the first component of the tuple), then the function call
f(\^{} *(3,"foo")) is type-correct and has type int.

Similar to ML, polymorphic functions are implicitly instantiated when
they are called.  They are not instantiated in any other context unless
done explicilty.

\subsubsection{Instantiation Expressions}

%-%\texttt{
%-%\begin{tabular}{lrl}
%-%<inst-exp> ::= <exp> @ <<type-exp>,...,<type-exp>>
%-%\end{tabular}
%-%}

\begin{verbatim}
<inst-exp> ::= <exp> @ <<type-exp>,...,<type-exp>>
\end{verbatim}

The expression must evaluate to a polymorphic function.  The function is
instantiated with the given type parameters.  For instance, if f is a
function that has type 'a $<$'a,'b$>$(*'a,'b), then the expression f @
$<$int,string$>$ has type int (int,string) (i.e., a function from an int
and a string to an int.)

\subsubsection{New Expressions}

%-%\texttt{
%-%\begin{tabular}{lrl}
%-%<new-exp> & ::= & <new>~<qid>[(<exp>,...,<exp>)]~|
%-%                  <new>~<qid>.<id>(<exp>)~|
%-%                  <new>~<qid>\{<id>=<exp>,...,<id>=<exp>\}~|
%-%                  <new>~*(<exp>,...,<exp>) \\
%-%<new>     & ::= & \textbf{new} | \^{} 
%-%\end{tabular}
%-%}

\begin{verbatim}
<new-exp> ::= <new> <qid>[(<exp>,...,<exp>)] | <new> <qid>.<id>(<exp>)
              <new> <qid>{<id>=<exp>,...,<id>=<exp>} |
              <new> *(<exp>,...,<exp>)

<new> ::= new | ^ 
\end{verbatim}


New expressions are used to create struct, tuple, union, exn, and
abstype values.  Either the keyword ``new'' or a carat (``\^{}'') may be
used to start the expression.  The former makes it easier to port Java
and C++ code, the latter is a lot less verbose.

To create a new struct, one may either use the struct name, followed by
a parenthesis-enclosed list of (unlabelled) expressions or else a
brace-enclosed list of expressions labelled by field name.   The former
implicitly associates the field names in the order in which they were
declared in the struct definition with the given expressions.  The
latter form ignores the order that the fields were declared in.  Both
forms evaluate the expressions left-to-right.  I generally use the
unlabelled form for small structs (e.g., lists) and the labelled form
for larger structs where I can't remember in which order the fields go
or where I feel like the reader may need more documentation.

Creating tuples is similar to creating structs with the unlabelled form.
The only difference is that one uses ``*'' instead of the name of the
struct.

For unions, new requires the name of the union followed by a period and
the name of the field (i.e., the data constructor).  If the field
carries void type, then no argument expression should be supplied.
Otherwise, if the field carries type T, then an argument expression of
type T should be applied.

\subsubsection{Dot Expressions}

%-%\texttt{
%-%\begin{tabular}{lrl}
%-%<dot-exp> & ::= & <exp>.<id>
%-%\end{tabular}
%-%}

\begin{verbatim}
<dot-exp> ::= <exp>.<id>
\end{verbatim}

A dot expression is used to extract a field from a struct or to
implicitly check that a union value is of a particular type.  In the
former case, $<$exp$>$ must be of a struct type with a field named by
$<$id$>$.  The type of the resulting expression is the type of the
associated field.  If the struct is an option-struct, then executing the
dot-expression will cause the NullPointerException to be raised when the
option-struct is null.  In the latter case, $<$exp$>$ must be of a union
type with a field named by $<$id$>$.  The resulting expression has the
type associated with the field in the union declaration.  When executed,
the tag on the union value is tested and the underlying value is
extracted and returned if the tag matches the field specified.
Otherwise, the UnionVariantException is raised.  We discourage the use
of dot-expressions for unions and encourage the use of switch statements
instead.   However, the dot-notation is particularly convenient for
porting C code.

\subsubsection{Subscript Expressions}

%-%\texttt{
%-%\begin{tabular}
%-%<subscript-exp> ::= <exp>[<exp>]
%-%\end{tabular}
%-%}

\begin{verbatim}
<subscript-exp> ::= <exp>[<exp>]
\end{verbatim}

The first expression must have an array type or be of type string, and
the second expression should have type unsigned int.  (Note that signed
integers are implicitly coerced to unsigned integers, and both shorts
and bytes are implicitly coerced to ints.)  If the second expression
evaluates to the integer i, then the ith value is extracted from the
array or string, as long as i is between 0 and the size of the array or
string.  Otherwise, the ArrayBoundsExcpetion is raised.  (Actually,
currently the program is terminated, but this should change soon.)

\subsubsection{Raise Expressions}

%-%\texttt{
%-%\begin{tabular}{lrl}
%-%<raise-exp> & ::= & \textbf{raise}~<qid>([<exp>])~|
%-%                    \textbf{raise}~(<exp>)
%-%\end{tabular}
%-%}

\begin{verbatim}
<raise-exp> ::= raise <qid>([<exp>]) | raise (<exp>)
\end{verbatim}

In the first syntax, the qualified identifier must be an exception name
in scope.  If the optional expression is present and has type T, then
the exception must have been declared to carry type T.  In the second
syntax, $<$exp$>$ must have type exn.  As in Java, the semantics of a
raise expression is to transfer control to the nearest (dynamically)
enclosing try/handle statement.  The type of the raise expression is
arbitrary.

In the future,  we are also likely to make the application of an
exception constructor a general expression form in which case, the first
syntactic form will be unecessary.  Note that for parsing reasons,
parentheses are required regardless as to whether an exception carries
values or not.

\subsubsection{Expression Sequences}

%-%\texttt{
%-%\begin{tabular}{lrl}
%-%<seq-exp> & ::= & <exp> , <exp>
%-%\end{tabular}
%-%}

\begin{verbatim}
<seq-exp> ::= <exp> , <exp>
\end{verbatim}

As in C, the first expression is evaluated and its result is discarded.
Then the second expression is evaluated and its result is returned as
the value of the compound expression.  It is often necessary to insert
parentheses around the expression to disambiguate the parsing in
contexts that also use commas as a separator (e.g., tuple expressions or
initialization arguments for structs or constant arrays.)

\subsubsection{Cast Expressions}

%-%\texttt{
%-%\begin{tabular}{lrl}
%-%<cast-exp> & ::= & (:<type-exp>)<exp>
%-%\end{tabular}
%-%}

\begin{verbatim}
<cast-exp> ::= (:<type-exp>)<exp>
\end{verbatim}

Cast expressions allow values of one type to be explicitly coerced to
another type.  Currently, casting is only supported on integral types.
We chose to require a colon before the type expression, unlike C, in
order to simplify parsing.

\subsubsection{Fun Expressions}

%-%\texttt{
%-%\begin{tabular}{lrl}
%-%<fun-exp> & ::= & \textbf{fun }<type-exp>
%-%                  <pid>[<<type-var>,...,<type-var>>] (<type-exp>
%-%                  <id>,...,<type-exp> <id>) <block>
%-%\end{tabular}
%-%}

% Odd indentation here, what's intended?
\begin{verbatim}
<fun-exp> ::= fun <type-exp> <pid>[<<type-var>,...,<type-var>>]
                    (<type-exp> <id>,...,<type-exp> <id>) <block>
\end{verbatim}

Fun expressions are a convenient way to define a static function within
the middle of some computation.  The syntax is pretty straightforward:
Just use the keyword ``fun'' followed by a function declaration as at
the top-level.  Unlike functional languages such as ML or Scheme, these
expressions do not close over their free variables.  That is, the
function cannot mention any local variables, type-variables, or
arguments of the enclosing function.  In essence, the function is
type-checked as if it was declared at the top-level.

There is a short-cut for declaring a variable of function type and
binding it to a function expression.  In particular, it is legal to
write a function declaration like:

%-%\texttt{
%-%\textbf{int} foo(\textbf{int} x,\textbf{string} y) \{ \textbf{return}
%-%    x+\textbf{ord}(y[3]); \};
%-%}

\begin{verbatim}
int foo(int x,string y) { return x+ord(y[3]); };
\end{verbatim}


within another function.  This is equivalent to the following local
declaration:

%-%\texttt{
%-%\textbf{int} foo(\textbf{int},\textbf{string}) = \textbf{fun}
%-%    foo(\textbf{int} x,\textbf{string} y) \{ \textbf{return}
%-%    x+\textbf{ord}(y[3]); \};
%-%}

\begin{verbatim}
int foo(int,string) = fun foo(int x,string y) { return x+ord(y[3]); };
\end{verbatim}


\subsection{Statements\label{statements}}

%-%\texttt{
%-%\begin{tabular}{lrl}
%-%<stmt> & ::= & <null-stmt> | <exp>; | <stmt> <stmt> | <if-stmt> |
%-%               <return-stmt>; | <while-stmt> | <do-stmt> | <for-smt> |
%-%               <break-stmt> | <continue-stmt> | <switch-stmt> |
%-%               <try-stmt> | <labelled-stmt> | <block-stmt> |
%-%               <printf-stmt>
%-%\end{tabular}
%-%}

\begin{verbatim}
<stmt> ::= <null-stmt> | <exp>; | <stmt> <stmt> | <if-stmt> |
           <return-stmt>; | <while-stmt> | <do-stmt> | <for-smt> |
           <break-stmt> | <continue-stmt> | <switch-stmt> | <try-stmt> |
           <labelled-stmt> | <block-stmt> | <printf-stmt>
\end{verbatim}

\subsubsection{Null Statement}

%-%\texttt{
%-%\begin{tabular}{lrl}
%-%<null-stmt> & ::= & ;
%-%\end{tabular}
%-%}

\begin{verbatim}
<null-stmt> ::= ;
\end{verbatim}

An empty (null, skip) statement is represented by a semi-colon.


\subsubsection{Expression Statements}

An expression may be used in any syntactic context for statements.   The
value of the expression is discarded.


\subsubsection{Compound Statements}

Two adjacent statements are executed in sequential order.


\subsubsection{Conditional Statements}

%-%\texttt{
%-%\begin{tabular}{lrl}
%-%<if-stmt> & ::= & \textbf{if} (<exp>) <stmt> | \textbf{if} (<exp>)
%-%                  <stmt> \textbf{else} <stmt>
%-%\end{tabular}
%-%}

\begin{verbatim}
<if-stmt> ::= if (<exp>) <stmt> | if (<exp>) <stmt> else <stmt>
\end{verbatim}

An if statement requires that the $<$exp$>$ argument have type bool.
The expression is evaluated and if it results in true, then the first
$<$stmt$>$ is executed.  In the first syntactic case, if the expression
evaluates to false, then the statement has no (additional) effect.  In
the second syntactic case, the second $<$stmt$>$ is executed.  Thus,
``if (e) s'' is equivalent to ``if (e) s else ;''.  Nested
if-expressions are disambiguated as in C.


\subsubsection{Return Statements}

%-%\texttt{
%-%\begin{tabular}{lrl}
%-%<return-stmt> & ::= & \textbf{return}; | \textbf{return} <exp>;
%-%\end{tabular}
%-%}

\begin{verbatim}
<return-stmt> ::= return; | return <exp>;
\end{verbatim}

Return is used to terminate execution of a function (immediately) and to
optionally return a value to the caller.  If the enclosing function has
result type void, then the first form must be used.  (Actually, for
obscure reasons we allow the second form provided that the exception has
type void.)  If the enclosing function has a type other than void, then
the second form must be used and $<$exp$>$ must have that type.   The
compiler attempts to ensure that along all control-flow paths within a
function, there is a return statement.  However, the analysis is fairly
simple and flow-insensitive so it is sometimes necessary for programmers
to insert extra return or raise statements even when the programmer is
sure that the return will never be executed.   In such situations, it is
usual to raise an exception indicating that a purportedly impossible
situation has arisen.

\subsubsection{While Statements}

%-%\texttt{
%-%\begin{tabular}{lrl}
%-%<while-stmt> & ::= & \textbf{while} (<exp>) <stmt>
%-%\end{tabular}
%-%}

\begin{verbatim}
<while-stmt> ::= while (<exp>) <stmt>
\end{verbatim}

The $<$exp$>$ must have type bool.  Execution of the statement proceeds
by first evaluating the expression.  If it evaluates to true, then the
body $<$stmt$>$ of the loop is executed, followed again by the entire
while expression.   Thus, the statement is roughly equivalent to ``if
($<$exp$>$) \{ $<$stmt$>$; while ($<$exp$>$) $<$stmt$>$ \}'' (ignoring
break and continue.)


\subsubsection{For Statements}

%-%\texttt{
%-%\begin{tabular}{lrl}
%-%<for-stmt> & ::= & \textbf{for}~([<exp>];~[<exp>];~[<exp>])~<stmt> |
%-%                   \textbf{for}~(<var-decl>~[<exp>];~[<exp>])~<stmt>
%-%\end{tabular}
%-%}

\begin{verbatim}
<for-stmt> ::= for ([<exp>] ; [<exp>]; [<exp>]) <stmt> |
               for (<var-decl> [<exp>]; [<exp>]) <stmt>
\end{verbatim}

For-statements are meant to be as similar to C as possible, while
admitting a certain convenience from Java for declaring a loop index
variable within the statement.  As in C, the first expression (or
variable declaration) is executed initially.  Then the second
expression, which must have type bool, is evaluated.   If the result is
false, then execution of the statement terminates.   Otherwise, the
statement body is executed followed by the third expression.   Then the
second test expression is evaluated again, followed if true by the body,
etc.  Thus, the for statement ``for (e1;e2;e3) s'' is equivalent to
``e1; while (e2) \{ s; e3 \}'' modulo break and continue issues.


\subsubsection{Do Statements}

%-%\texttt{
%-%\begin{tabular}{lrl}
%-%<do-stmt> & ::= & \textbf{do} <stmt> \textbf{while} (<exp>)
%-%\end{tabular}
%-%}

\begin{verbatim}
<do-stmt> ::= do <stmt> while (<exp>)
\end{verbatim}

Similar to the while statement, except that the body of the loop is
executed before the test expression.  Thus, ignoring break and continue,
the construct is equivalent to ``$<$stmt$>$; while (exp) $<$stmt$>$''.


\subsubsection{Break Statements}

%-%\texttt{
%-%\begin{tabular}{lrl}
%-%<break-stmt> & ::= & \textbf{break}; | \textbf{break} <id>;
%-%\end{tabular}
%-%}

\begin{verbatim}
<break-stmt> ::= break; | break <id>;
\end{verbatim}

Execution of a simple break statement causes control to be transferred
out of the nearest lexically enclosing while, for, or do loop.  Control
is transferred to the statement immediately succeeding the enclosing
loop.  If the break does not occur within a loop, then an error will be
signalled at compile time.  Note that break is \textit{not} used to
transfer control out of switch statements as is done in C, C++, and
Java.

Execution of break $<$id$>$ causes control to be transferred to the
statement labelled with the identifier $<$id$>$.  Control may only be
transferred within the same block or to an outer block.  That is, it is
impossible to jump into a block (which may potentially define new
variables).


\subsubsection{Continue Statements}

%-%\texttt{
%-%\begin{tabular}{lrl}
%-%<continue> ::= \textbf{continue}; | \textbf{continue} <id>;
%-%\end{tabular}
%-%}

\begin{verbatim}
<continue> ::= continue; | continue <id>;
\end{verbatim}

Execution of a simple continue statement causes control to be
transferred to the ``test'' portion of the nearest lexically enclosing
loop.  If the continue does not occur within a loop, then an error will
be signalled at compile time.

Execution of continue $<$id$>$ causes control to be transferred to the
statement labelled with the identifier $<$id$>$ and has restrictions
similar to that of break $<$id$>$.  In fact, as I'm writing this, I'm
not sure why we provide both labelled break and continue instead of just
goto and expect that Fred will clear this up for me.


\subsubsection{Switch Statements}

%-%\texttt{
%-%\begin{tabular}{lrl}
%-%<switch-stmt>  & ::= & <int-switch> | <char-switch> | <union-switch> |
%-%                       <exn-switch> \\
%-%<int-switch>   & ::= &
%-%    \begin{tabbing}
%-%    \textbf{switch} (<exp>) \{ \=\textbf{case} <num>: <stmt> ... \\
%-%                               \>\textbf{case} <num>: <stmt> \\
%-%                               \>\textbf{default}: <stmt> \}
%-%    \end{tabbing} \\
%-%<char-switch>  & ::= &
%-%    \begin{tabbing}
%-%    \textbf{switch} (<exp>) \{ \=\textbf{case} <char>: <stmt> ... \\
%-%                               \>\textbf{case} <char>: <stmt> \\
%-%                               \>\textbf{default}: <stmt> \}
%-%    \end{tabbing} \\
%-%<union-switch> & ::= &
%-%    \begin{tabbing}
%-%    \textbf{switch} (<exp>) \{ \=\textbf{case} <id>[<pat>]: <stmt> ...
%-%        \\
%-%                               \>\textbf{case} <id>[<pat>]: <stmt> \\
%-%                               \>[\textbf{default}: <stmt>] \}
%-%    \end{tabbing} \\
%-%<exn-switch>   & ::= &
%-%    \begin{tabbing}
%-%    \textbf{switch} (<exp>) \{ \= \textbf{case} <qid>[<pat>]: <stmt> ...
%-%        \\
%-%                               \>\textbf{case} <qid>[<pat>]: <stmt> \\
%-%                               \>\textbf{default}: <stmt> \}
%-%    \end{tabbing} \\
%-%<pat>          & ::= & (~<prim-pat>~) | *(~<prim-pat>,...,<prim-pat>~) \\
%-%<prim-pat>     & ::= & {\_} | <id>
%-%\end{tabular}
%-%}

\begin{verbatim}
<switch-stmt> ::= <int-switch> | <char-switch> | 
                  <union-switch> | <exn-switch>
<int-switch> ::= switch (<exp>) { case <num>: <stmt> ... 
                                  case <num>: <stmt>
                                  default: <stmt> }
<char-switch> ::= switch (<exp>) { case <char>: <stmt> ...
                                   case <char>: <stmt> 
                                   default: <stmt> }
<union-switch> ::= switch (<exp>) { case <id>[<pat>]: <stmt> ...
                                    case <id>[<pat>]: <stmt>
                                    [default: <stmt>] }
<exn-switch> ::= switch (<exp>) { case <qid>[<pat>]: <stmt> ...
                                  case <qid>[<pat>]: <stmt>
                                  default: <stmt> }
<pat> ::= ( <prim-pat> ) | *( <prim-pat>,...,<prim-pat> )
<prim-pat> ::= _ | <id> 
\end{verbatim}

There are four different kinds of switch expressions.  The
$<$int-switch$>$ is used to test integers, $<$char-switch$>$ to test
characters, $<$union-switch$>$ to test values of union type and to
extract the values carried by a data constructor, and the
$<$exn-switch$>$ to test and extract values carried by an exception
constructor.  The syntax for all four kinds of switches is quite similar
and the semantics is close to, but subtly different than C.

In particular, \textit{switch does not support implicit fall-through
from one case to the next.}  Thus, it is as if an implicit C ``break''
is inserted before each case or default label.  Furthermore, a break
statement within a switch does not cause control to be transferred out
of the nearest lexically enclosing switch, but rather the nearest
lexically enclosing loop.  This was always a brain-damaged part of C
(witness the AT\&T switch crash) and we are happy to drop it.  In
addition, as explained in Section [?], cases for union switches and
exception switches introduce bound variables whose scope is the
associated case's $<$stmt$>$.   Supporting implicit fall-through would
thus allow a statement to access an undefined variable.

In all of the switch constructs, it is necessary for the cases to be
disjoint and complete.  This constraint is enforced by the type-checker.
A switch is disjoint is no case matches the same integer, character,
union field, or exception constructor.  A switch is complete if every
integer, character, union field, or exception has a case.  (Note, in
fact one can leave the default off of an exception switch and the
type-checker inserts a default case which raises the exception being
tested.)

Union switches allow a primitive form of pattern matching in the style
of ML.  In particular, if the argument union expression evaluates to a
value with a tag corresponding to field f, then the case for f will be
selected when the switch is executed.  If f carries type void, then it
is an error for the case to define the optional pattern.  If the field
carries any type T besides void, then the case must include the optional
pattern.  The pattern can either be a wild-card (underscore), a
variable, or a tuple pattern.  Wild-card and variable patterns can be
used for any type, but tuple-patterns may only be used for tuple-types.
A tuple pattern provides either wild-card or variable sub-patterns for
binding the components of the tuple.  Within the scope of the case's
associated statement, the variables of the pattern (if any) are bound to
the appropriate values.

Exception switches are similar to union switches except that instead of
matching on fields within a union, the switch takes an exn value and
matches cases with exception constructor names.  As with unions, if an
exception carries no type, then no optional pattern should be supplied.
Otherwise, if the exception is declared to carray values of type T, then
an optional pattern should be supplied.

Note that integer and character cases require numbers or characters,
which is slightly more restrictive than C which allows some expressions
such as ``2+3''.  In the future, we expect to add more support for
pattern matching, ``or-patterns'', ranges for character and integer
switches, etc.

\subsubsection{Try Statements}

%-%\texttt{
%-%\begin{tabular}{lrl}
%-%<try-stmt> & ::= & \textbf{try} <stmt> \textbf{handle} <id> <stmt> | \\
%-%           &     & \begin{tabbing}
%-%                   \textbf{try} <stmt> \textbf{catch} \{
%-%                       \=\textbf{case} <qid>[<pat>]: <stmt> ... \\
%-%                       \>\textbf{case <}qid>[<pat>]: <stmt> \\
%-%                       \>[\textbf{default: }<stmt>] \}
%-%                   \end{tabbing}
%-%\end{tabular}
%-%}

\begin{verbatim}
<try-stmt> ::= try <stmt> handle <id> <stmt> | 
               try <stmt> catch { case <qid>[<pat>]: <stmt> ...
                                  case <qid>[<pat>]: <stmt>
                                  [default: <stmt>] }
\end{verbatim}

In both forms, execution proceeds by evaluating the first statement.  If
this does not cause an uncaught exception to be raised, then the
statement terminates without executing the second clause.  If an
exception is raised during evaluation of the first statement and is not
caught by an intervening (dynamic) handler, then control is transferred
to the second clause.  In the case of a try/handle statement, the
exception value of type exn passed to raise is bound to the identifier
$<$id$>$ within the scope of the second statement.   Typically, the second
statement performs an exception switch on the identifier to determine
what exception was raised and to take appropriate action.  This pattern
is so frequent that we provide the more convenenient ``catch'' form as
in Java which allows one to immediately start switching on the exception
without having to name it.

In the future, we expect to drop the try/handle form and to add
try/finally and try/catch/finally similar to Java.


\subsubsection{Labelled Statements}

%-%\texttt{
%-%\begin{tabular}{lrl}
%-%<label-stmt> & ::= & <id> : <stmt>
%-%\end{tabular}
%-%}

\begin{verbatim}
<label-stmt> ::= <id> : <stmt>
\end{verbatim}

Labels are placed on statements to support easy break and continue to a
specified location.


\subsubsection{Block Statements}

%-%\texttt{
%-%\begin{tabular}{lrl}
%-%<block-stmt> & ::= & \{ <var-decl> ... <var-decl> <stmt> \}
%-%\end{tabular}
%-%}

\begin{verbatim}
<block-stmt> ::= { <var-decl> ... <var-decl> <stmt> }
\end{verbatim}

A block statement supports both nested variable declarations and a way
to disambiguate the parsing of certain statement patterns (e.g., the
dangling else problem with if.)   Control may only be transferred into
the beginning of a block (see break and continue).


\subsubsection{Printf and Fprintf Statements and Sprintf Expressions}

%-%\texttt{
%-%\begin{tabular}{lrl}
%-%<printf-stmt> & ::= & \textbf{printf}(<desc>,<exp>,...<exp>) | 
%-%                      \textbf{fprintf}(<exp>,<desc>,<exp>,...,<exp>) \\
%-%\\
%-%<sprintf-exp> & ::= & \textbf{sprintf}(<desc>,<exp>,...<exp>)
%-%\end{tabular}
%-%}

\begin{verbatim}
<printf-stmt> ::= printf(<desc>,<exp>,...<exp>) | 
                  fprintf(<exp>,<desc>,<exp>,...,<exp>)

<sprintf-exp> ::= sprintf(<desc>,<exp>,...<exp>)
\end{verbatim}

The printf and fprintf statements are used for printing integers,
strings, and characters to files or terminals, similar to the printf and
frpintf functions of the C standard library.  The sprintf expression is
similar but produces a string as a result instead of writing to some
kind of stream.  The $<$desc$>$ argument must be a string literal and is
parsed at compile time to determine the appropriate number and types of
the additional arguments to printf.   In the case of fprintf, the first
argument expression is meant to be a FILE value such as tal{\_}stdout.
Descriptors follow the conventions of C.  So for instance, ``\%d'' is
used for signed integers, ``\%u'' for unsigned, ``\%x'' for lower-case
hex, ``\%s'' for strings, and ``\%c'' for characters. Currently, there
is no support for precision in the output, other types, etc.   These
statements were included purely to make porting C code easier and to
simplify mundane output tasks.


\subsection{Program Structure\label{programs}}


\subsection{BNF Grammar}


\section{The Popcorn Library\label{library}}

For now, see the header files in the library directory of the
distribution.  Note that Core exports some types and values that are
actually defined in the runtime, such as the FILE type.


\subsection{Core}

For new{\_}array, the first expression should evaluate to an unsigned
integer indicating the size of the array, and the second expression
should be a value of the array type.  This value is used to initialize
the contents of the array.

For new{\_}string, the argument should evaluat to an unsigned integer
indicating the desired size of the string.  The string is filled with
unspecified values.


\subsection{Id}


\subsection{Set}


\subsection{List}


\subsection{Dict}


\subsection{Queue}


\subsection{Set}



\section{The Popcorn Tools\label{tools}}


\subsection{The popcorn Compiler}


\subsection{The talc Type Checker and Assembler}


\subsection{The poplex Lexer Generator\label{lex}}

The developers of OCaml graciously allowed us to retarget their lexer
generator to Pocporn.  That is, popocamllex is an ocaml program that
given a lexer specification, produces a Popcorn file.  It is a modified
version of the ocamllex program that we did not write.   Here is how to
use popocamllex.

Use popocamllex exactly as you would use ocamllex except: 

\begin{itemize}

\item Give your files extension .popl

\item Result file will be .pop

\item Make your header, trailer, and actions Popcorn code.

\item // style comments aren't supported so be careful with them
(basically don't put troubling characters like unmatched braces and " in
them).  /*~*/ style  should be supported.

\item Actions should \textit{return} a value of type int.  Yes, you
should use the Popcorn keyword return.

\item Files lexing.pop and lexing.h are in the Popcorn standard library
and must be used with all generated lexers.  This recipe works:

    \begin{enumerate}
    \item Put ``\#include "lexing.h"'' in your .popl file.
    \item Link in lexing.pop.
    \end{enumerate}

\end{itemize}

Note that all of the lex-specific syntax (for rules, patters, etc) is as
in ocamllex.

Interfacing with popbison is exceptionally tricky since the two programs
are not really on the same wavelength.  This should help: 

\begin{enumerate}

\item Have some main function create a lexbuf by calling the
from{\_}function function in the lexing module.  Put the result in a
global variable (use an Opt so it can be initialized).

\item Write a yylex() function which calls a lexer entry point with the
lexbuf.  (Hint: each rule is translated to a function of the same
name.)

\item Bison creates a header file you'll want to include.  Note this
header file may refer to other types that it doesn't define -- you
currently must extern them by other means.

\item In your actions, put values of type YYSTYPE in yylval.

\item Define and manipulate int yyline and void yyerror(string)
appropriately.

\item Make sure the eof actions return a negative integer or you will
get parse errors.

\end{enumerate}


\subsection{The popbison Parser Generator\label{bison}}

We have modified the publicly available Bison program to produce parsers
written in Popcorn.  That is, popbison is a C program that given a
parser specification, produces a Popcorn file.  It is a modified version
of the Bison program which we did not write.  Here is how to use
popbison.

Use popbison exactly as you would bison except: 

\begin{itemize}

\item Give your files extension .y

\item Result files be {\_}tab.pop and {\_}tab.h

\item Make your header, trailer, and actions Popcorn code.

\item Use the \%union, and \^{} \$ notation as described below.

\end{itemize}

The .h file is suitable for including within a lexer or some other tool.
However, it does not include any other header files, so if it mentions
types from other modules, you will have to include them by other means.
The parser contains all of the tables as well as the parse engine.  It's
invoked by calling yyparse() which in turn calls yylex() expecting that
yylex() produces an integer token and a semantic value in yylval.

The type of yylval is specified with the \texttt{\%union} declaration.
This is \textbf{required} for us (but optional for C-bison.)  The syntax
for the union declaration does not mention the type -- it becomes
YYSTYPE.  Note that this union should have as its initial field
something that can be used at top level for initialization.  For
instance, the union should not be polymorphic and the first field should
carry a type (such as int, bool, void, ?struct, etc.) that has a default
value.

One declares tokens using the \%token $<$id$>$ notation as in bison.

One declares the type of tokens (terminals) and non-terminals using
\%type $<$field$>$ t1 t2 ... tn where $<$field$>$ is a field of the
union type, and t1,t2,...,tn are [non-]terminals.

The grammar rules work just as in bison/yacc.  The biggest difference is
within the actions of the grammar rules.  In particular, \$\$ refers to
yylval.  Because it's declared to be of type YYSTYPE, anything that goes
into \$\$ should have this type.  If you use it as an r-value, then you
probably need to switch on it or use the new union projection (e.g.,
\$\$.foo).

\$1 \$2 and so forth continue to refer to the semantic elements of the
rule. They are translated as (yysv[yysvp{\_}offset-i]).$<$field$.$ where
$<$field$.$ is associated with that particular terminal/non-terminal in
a \%type declaration. That is, a reference to \$1 will implicitly
``downcast'' to the ``type'' of the [non-]terminal.

Some new syntax is \^{} \$.  This is essentially translated to
\^{} YYSTYPE.$<$field$>$ where $<$field$>$ is the field name of the union
associated with \$\$ (i.e., the rule's left-hand-side.)

So for instance, in the grammar below, for the ``+'' case, we have as
the action:

%-%\texttt{
%-%exp : exp exp '+' \{ \$\$ = \^{} \$(\$1 + \$2); \}
%-%}

\begin{verbatim}
      exp : exp exp '+' { $$ = ^$($1 + $2); } 
\end{verbatim}

The \$1 gets the exp value.  Since exp is associated with the foo label
of the union, this means that the semantic value is cast to ``int'' from
YYSTYPE.  Similarly for \$2.  The default semantic value would be there
for the token '+'.  Anyway, because \$1 and \$2 are automatically cast
to integers, we can add them to produce an integer.  But we can't just
place the result in \$\$ -- rather, we have to ``cast up'', hence the
\^{} \$ wrapped around the expression.

%-%\texttt{
%-%file rpcalc.y
%-%
%-%\begin{tabular}{|l|}
%-%\hline
%-%\%\{ \\
%-%extern void print{\_}string(string); \\
%-%extern int print{\_}int(int); \\
%-%\%\} \\
%-%\\
%-%\%union \{ \\
%-%~~int    foo; \\
%-%~~short    x; \\
%-%\}
%-%\\
%-%\%token NUM \\
%-%\%type <foo> exp NUM \\
%-%\%type <x> line input \\
%-%\\
%-%\%\% /* Grammar rules and actions follow */ \\
%-%\\
%-%\begin{tabbing}
%-%~~~~~~~~~~\=~~~~~~~~~~~~~~~~~~~~\= \kill
%-%input:    \>/*empty*/           \>\{ \$\$ = \^{} \$(3); \} \\
%-%          \>| input line        \>\{ \$\$ = \^{} \$(4); \} \\
%-%;\\
%-%\\
%-%line:     \>'$\backslash$n'     \>\{ \$\$ = \^{} \$(6); \} \\
%-%          \> | exp '$\backslash$n' \>\{ print{\_}string ("RESULT="); \\
%-%          \>                    \>print{\_}int(\$1); \\
%-%          \>                    \>\$\$ = \^{} \$(5); \} \\
%-%;\\
%-%\\
%-%exp:      \>NUM                 \>\{ \$\$ = \^{} \$(\$1); \} \\
%-%          \>| exp exp '+'       \>\{ \$\$ = \^{} \$(\$1 + \$2); \} \\
%-%          \>| exp exp '-'       \>\{ \$\$ = \^{} \$(\$1 - \$2); \} \\
%-%          \>| exp exp '*'       \>\{ \$\$ = \^{} \$(\$1 - \$2); \} \\
%-%;\\
%-%\%\%
%-%\end{tabbing}
%-%
%-%\\ \hline
%-%\end{tabular}
%-%}

\begin{verbatim}
---file
rpcalc.y-------------------------------------------------------------- 
%{ 
extern void print_string(string); 
extern int print_int(int); 
%} 

%union { 
  int    foo; 
  short    x; 
} 

%token NUM 
%type <foo> exp NUM 
%type <x> line input 

%% /* Grammar rules and actions follow */ 

input:   /*empty*/      { $$ = ^$(3); } 
        | input line   { $$ = ^$(4); } 
; 

line:   '\n'       { $$ = ^$(6); } 
        | exp '\n' { print_string ("RESULT="); 
                     print_int($1); 
                     $$ = ^$(5); } 
; 

exp:    NUM { $$ = ^$($1); } 
        | exp exp '+' { $$ = ^$($1 + $2); } 
        | exp exp '-' { $$ = ^$($1 - $2); } 
        | exp exp '*' { $$ = ^$($1 - $2); } 

; 
%% 
\end{verbatim}

%\printindex

\end{document}
